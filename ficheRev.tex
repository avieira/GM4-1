\documentclass{article}
\input{../preambule}
%\usepackage[8pt]{extsizes}

\hypersetup{colorlinks=true, urlcolor=bleu, linkcolor=red}

%Def = Definition
%Theo = Théorème
%Prop = Propriété
%Coro = Corollaire
%Lem = Lemme

\makeatletter
\@addtoreset{section}{part}
\makeatother

\begin{document}

\setcounter{tocdepth}{4}
\tableofcontents
\newpage

\part{Automatique}
\begin{enumerate}
\item Définition linéaire, invariant.
\item Définition accessible, ensemble d'accessibilité, contrôlable, complètement contrôlable.
\item Solution de l'équation $\dot{x}(t)=Ax(t)+Bu(t)$
\item Définition transformée de Laplace. Transformée de $f'$. Transformée de cos, sin et exponentielle.
\item Fonction de transfert
\end{enumerate}

\part{\'El\'ements finis}
\begin{enumerate}
\item Intégration par parties
\item Formule de Stokes
\item Formule d'Ostrogradsky
\item Première et deuxième identité de Green
\item Conditions de Dirichlet
\item Théorème de Lax-Milgram $\sim$ Démonstration...
\item Choix de $V_h$, dimension ?
\item Fonctions de base, problème discrétisé
\item Matrice de rigidité
\item \textbf{R est définie positive}
\item Espace de Hilbert (TOUT définir)
\item Définition de $\mathcal{D}(\Omega)$ et de $\mathcal{D}'(\Omega)$. Quid que $\Omega$ ?
\item Pseudo-topologie sur $\mathcal{D}(\Omega)$ et sur $\mathcal{D}'(\Omega)$
\item Dérivation des distributions. Continuité ?
\item Définition de $\partial^\alpha$
\item Définition de $L^p(\Omega)$. Produit scalaire et norme sur $L^2(\Omega)$. 
\item Lien entre $\mathcal{D}(\Omega)$ et $L^2(\Omega)$. 
\item Densité de $\mathcal{D}(\Omega)$. 
\item Injection canonique
\item Lien entre la convergence dans $L^2(\Omega)$ et celle dans $\mathcal{D}'(\Omega)$
\item Définition de $H^1(\Omega)$ et $H_0^1(\Omega)$
\item Produit scalaire et norme sur $H^1(\Omega)$
\item $H^1(\Omega)$ : 3 propriétés. \textbf{Il est complet}
\item Propriété de Rellich. Application compacte
\item $H^1_0(\Omega)$ : Propriété.
\item \textbf{Formule de Poincaré}
\item \textbf{Semi norme sur $H_0^1(\Omega)$ et lien avec $\|\bullet\|_{H^1(\Omega)}$}
\item Définition de $\gamma_0$
\item Définition de $W^{m,p}(\Omega)$. Norme sur cet espace.
\item Fonctions $\mu$-Holderienne
\item Condition d'injection canonique entre $H^m(\Omega)$ et $\mathcal{C}^s(\Omega)$
\item Norme sur un dual
\item Théorème de projection
\item Théorème de représentation de Riesz-Fréchet
\item \textbf{Théorème de Stanpaccia}
\item \textbf{Lemme de Céa}
\item Qu'est-ce qu'un élément fini ?
\item Définition d'unisolvance. Comment la démontrer ?
\end{enumerate}

\part{Béziers-Splines}
\begin{enumerate}
\section{Courbes paramétrées}
\item Régularité, p-régulier
\item Arc admet un veteur limite tangeant
\item Suivant p et q pairs et impairs
\item Branches infinies dans le cas des courbes planes
\item Revoir réduction d'intervalle
\item Longueur de l'arc
\item Arcs équivlents
\item Courbes gauche : tangente, plan normal, plan osculateur, normale principale
\item Paramétrisation normale, abscisse curviligne
\item Exemple de paramètre admissible
\item Courbure algébrique, rayon de courbure, centre de courbure, cercle osculateur, développée
\item Formules de Frénet
\item Trouver un plan tangeant
\section{Splines}
\item Définition de la suite $\tau$, $r$, $\mathcal{P}^{k,\tau,r}$
\item Dimension et base de $\mathcal{P}^{k,\tau,r}$
\item Particularité des fonctions splines, dimension de l'espace des fonctions splines
\item Conditions (C) pour splines cubiques
\item Unicité ?
\item Problème de minimisation vérifié par une fonction spline cubique
\section{B-Splines}
\item Définition de n\oe uds, multiplicité k
\item Définition de $w_{ij}(x)$
\item Relation de récurrence pour les $B_{i,k}$
\item Quelle multiplicité pour avoir $B_{i,k}$ nul ?
\item 6 propriétés des $B_{i,k}$
\item Formule de dérivée à droite
\item Définition de $\mathcal{P}^{k,\tau}$
\item Lien entre n, m et k. Condition pour une base.
\section{Algorithme}
\item Définition de $S$
\item Algorithme de De Casteljan dévaluation en un point
\item Algorithme des dérivées
\item Algorithme d'insertion d'un n\oe ud
\end{enumerate}

\part{Optimisation linéaire}
\begin{enumerate}
\item Définition infimum, minimum
\item Définition coercive
\item Deux exemples fonctions coercives
\item \textbf{Exemple de J coercive}
\item \textbf{Condition pour que J atteigne son minimum}
\item \textbf{Rapport frontière et minimum}
\item Définition dérivée directionnelle, Gâteaux-différentiable, gradient
\item Définition de Fréchet-différentiable
\item \textbf{Fréchet $\Rightarrow$ Gâteaux}
\item Définition espace convexe, épigraphe.
\item \textbf{Équivalence fonction convexe}
\item Définition strictement convexe, $\alpha$-convexe.
\item Équivalence à $f$ convexe
\item \textbf{Équivalence à $\alpha$-convexe}
\item \textbf{Condition d'optimalité dans un ouvert}
\item \textbf{Condition nécessaire puis condition suffisante pour un minimum}
\item \textbf{Théorème de Kuhn et Tucker}
\item Contraintes qualifiées
\item \textbf{Théorème dans le cas des contraintes qualifiées}
\item \textbf{Condition nécessaire de qualification}
\item \textbf{Lemme : équivalence à un minimum}
\item \textbf{Équivalent à minimum avec $\lambda$}
\item Point selle
\item \textbf{Propriété des points selles : système vérifié}
\item \textbf{Lemme : inégalité sup et inf}
\item \textbf{Problème dual}
\item Problème d'optimisation linéaire, passage du problème sous forme canonique à la forme standard
\item Ensemble des solutions réalisables, sommets
\item Définition de $\Gamma$, $A_{\gamma}$ et $\mathcal{B}$. 
\end{enumerate}

\part{Processus de Markov}
\begin{enumerate}
\item Définition de processus
\item Propriété de Markov, homogénéité
\item Mesure de probabilité, $f:E\to\mathbb{C}$ : représentation vectorielle
\item Matrice stochastique
\item \textbf{Relations de Kolmogorov}
\item Définition "$i$ conduit à $j$", conduit ) un préordre. Notation.
\item $i$ et $j$ communiquent, relation d'équivalence, notation.
\item Définition transitive, finale, ergodiques.
\item Existence de classes finales dans le cas fini
\item Forme canonique matrice de transition, puissance n
\item \textbf{Si E fini, alors le processus finira presque surement dans une des classes finales}
\item \textbf{A quoi correspond $(I-Q)^{-1}$ ?}
\item \textbf{B=NR ?}
\item Ensemble des entiers avec un chemin
\item Propriété fondamentale
\item \textbf{PGCD($N_{ii}$)}
\item Période d'une classe, classe apériodique.
\item \textbf{Forme de $N_{ii}$ et de $N_{ij}$}
\item Définition de $i\sim j$
\item Nombre de sous-classes cycliques
\item Chaîne régulière
\item \textbf{Équilvalence à chaîne régulière}
\item \textbf{Théorème fondamental des chaînes régulières}
\item \textbf{Théorème ergodique}
\end{enumerate}

\part{Statistiques}
\begin{enumerate}
\item Définition suite chronologique
\item Deux modèles de décomposition : quelle combinaison ? Dans quels cas va-t-on choisir l'un ou l'autre ?
\item Moindres carrés : expression de $a$ et $b$. 
\item Mise en place de la méthode des deux points
\item Expression de $r$
\item Moindre carré polynomial : expression de $\theta^{MC}$
\item Formule des moyennes mobiles
\item Définition MMC
\item Étape dans la décomposition d'une série chronologique
\item Loi du chi2 : espérance et variance
\item Loi de Student, de Fisher, carré d'une Student
\item Définition MLG
\item Définition des estimateurs A, B et $\sigma^2$. Lois de chacun.
\item $d_x^2$ ?
\item Statistiques pour avoir les intervalles de confiance pour $\alpha$, $\beta$ et $\sigma^2$
\item IC de chacun d'entre eux
\item Test significatif du lien linéaire : Stat et zone de rejet
\item Test d'un modèle linéaire spécifique
\item IC pour $\mathbb{E}(Y_0)$ et pour une observation $Y_0$
\item Test du caractère significatif de la liaison linéaire par comparaison de modèle
\item ANOVA 1 : Gueule des données et du modèle
\item Definition dimension
\item Définition de $M_1$ et $M_p$
\item Estimation des paramètres dans chacun des modèles
\item Statistique de test
\item Définition contraste
\item Statistique de test pour les contrastes
\item Estimateur des $\mu_i$ et des $\sigma_i^2$
\item IC pour $\mu_i$
\end{enumerate}
\end{document}
