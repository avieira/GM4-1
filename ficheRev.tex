\documentclass{article}
\input{../preambule}
%\usepackage[8pt]{extsizes}

\hypersetup{colorlinks=true, urlcolor=bleu, linkcolor=red}

%Def = Definition
%Theo = Théorème
%Prop = Propriété
%Coro = Corollaire
%Lem = Lemme

\makeatletter
\@addtoreset{section}{part}
\makeatother

\begin{document}

\setcounter{tocdepth}{4}
\tableofcontents
\newpage

\part{Automatique}
\begin{enumerate}
\item Définition linéaire, invariant.
\item Définition accessible, ensemble d'accessibilité, contrôlable, complètement contrôlable.
\item Solution de l'équation $\dot{x}(t)=Ax(t)+Bu(t)$
\item Définition transformée de Laplace. Transformée de $f'$. Transformée de cos, sin et exponentielle.
\item Fonction de transfert
\item Effet d'un changement de base
\item Définition observable, observateur
\item Définition stabilité, point d'équilibre
\item Lemme : expression de l'exponentielle de $At$
\item Théorème de Kalman sur la controlabilité
\item Propriété sur la controlabilité
\item Définition complètement observable
\item Théorème de Kalman sur l'observabilité
\item Observateur de Luenberger
\item Critère des valeurs propres : matrice définie par bloc, matrice de transfert
\item Stabilité avec valeur propre
\item Lemme de Lyapounov
\item Trouver l'entrée pour stabiliser asymptotiquement S à l'origine
\item Système non linéaire : définition du linéarisé, de $\varepsilon$
\item Définition de pseudo-linéaire
\item Définition de point d'équilibre critique
\item Théorème de Hartman
\item Définition système affine
\item Définition du crochet, de $\Delta(z)$
\item Condition nécessaire pour qu'un système affine soit contrôlable à partir de $z\in\mathbb{R}^n$
\item Ensemble d'accessibilité depuis $z$
\item Proposition avec la condition du rang
\item Théorème sur E compact avec un système affine
\item Si aucune fonction F, que dire pour la comtrôlabilité ?
\item Condition nécessaire et suffisante pour la contrôlabilité complète d'un système affine complet.
\item Définition d'une fonction d'énergie de S
\item Définition d'une fonction propre
\item Définition système conservatif ou dissipatif
\item Commande feedback qui stabilise un système
\item Théorème sur la stabilisation assyptotique d'un système affine conservatif
\item Définition découplé
\end{enumerate}

\part{\'El\'ements finis}
\begin{enumerate}
\item Intégration par parties
\item Formule de Stokes
\item Formule d'Ostrogradsky
\item Première et deuxième identité de Green
\item Conditions de Dirichlet
\item Théorème de Lax-Milgram $\sim$ Démonstration...
\item Choix de $V_h$, dimension ?
\item Fonctions de base, problème discrétisé
\item Matrice de rigidité
\item \textbf{R est définie positive}
\item Espace de Hilbert (TOUT définir)
\item Définition de $\mathcal{D}(\Omega)$ et de $\mathcal{D}'(\Omega)$. Quid que $\Omega$ ?
\item Pseudo-topologie sur $\mathcal{D}(\Omega)$ et sur $\mathcal{D}'(\Omega)$
\item Dérivation des distributions. Continuité ?
\item Définition de $\partial^\alpha$
\item Définition de $L^p(\Omega)$. Produit scalaire et norme sur $L^2(\Omega)$. 
\item Lien entre $\mathcal{D}(\Omega)$ et $L^2(\Omega)$. 
\item Densité de $\mathcal{D}(\Omega)$. 
\item Injection canonique
\item Lien entre la convergence dans $L^2(\Omega)$ et celle dans $\mathcal{D}'(\Omega)$
\item Définition de $H^1(\Omega)$ et $H_0^1(\Omega)$
\item Produit scalaire et norme sur $H^1(\Omega)$
\item $H^1(\Omega)$ : 3 propriétés. \textbf{Il est complet}
\item Propriété de Rellich. Application compacte
\item $H^1_0(\Omega)$ : Propriété.
\item \textbf{Formule de Poincaré}
\item \textbf{Semi norme sur $H_0^1(\Omega)$ et lien avec $\|\bullet\|_{H^1(\Omega)}$}
\item Définition de $\gamma_0$
\item Définition de $W^{m,p}(\Omega)$. Norme sur cet espace.
\item Fonctions $\mu$-Holderienne
\item Condition d'injection canonique entre $H^m(\Omega)$ et $\mathcal{C}^s(\Omega)$
\item Norme sur un dual
\item Théorème de projection
\item Théorème de représentation de Riesz-Fréchet
\item \textbf{Théorème de Stanpaccia}
\item \textbf{Lemme de Céa}
\item Qu'est-ce qu'un élément fini ?
\item Définition d'unisolvance. Comment la démontrer ?
\end{enumerate}

\part{Béziers-Splines}
\begin{enumerate}
\section{Approximation de courbes dans $\mathbb{R}$}
\subsection{Courbes paramétrées}
\item Régularité, p-régulier
\item Arc admet un veteur limite tangeant
\item Suivant p et q pairs et impairs
\item Branches infinies dans le cas des courbes planes
\item Revoir réduction d'intervalle
\item Longueur de l'arc
\item Arcs équivlents
\item Courbes gauche : tangente, plan normal, plan osculateur, normale principale
\item Paramétrisation normale, abscisse curviligne
\item Exemple de paramètre admissible
\item Courbure algébrique, rayon de courbure, centre de courbure, cercle osculateur, développée
\item Formules de Frénet
\item Trouver un plan tangeant
\subsection{Splines}
\item Définition de la suite $\tau$, $r$, $\mathcal{P}^{k,\tau,r}$
\item Dimension et base de $\mathcal{P}^{k,\tau,r}$
\item Particularité des fonctions splines, dimension de l'espace des fonctions splines
\item Conditions (C) pour splines cubiques
\item Unicité ?
\item Problème de minimisation vérifié par une fonction spline cubique
\subsection{B-Splines}
\item Définition de n\oe uds, multiplicité k
\item Définition de $w_{ij}(x)$
\item Relation de récurrence pour les $B_{i,k}$
\item Quelle multiplicité pour avoir $B_{i,k}$ nul ?
\item 6 propriétés des $B_{i,k}$
\item Formule de dérivée à droite
\item Définition de $\mathcal{P}^{k,\tau}$
\item Lien entre n, m et k. Condition pour une base.
\subsection{Algorithme}
\item Définition de $S$
\item Algorithme de De Casteljan dévaluation en un point
\item Algorithme des dérivées
\item Algorithme d'insertion d'un n\oe ud
\section{Approximation de courbes dans $\mathbb{R}^s$}
\subsection{Courbes B-Splines}
\item Définition des polynômes de Bernstein, expression explicite
\item 5 propositions
\item Définition d'une courbe B-Spline, point de contrôle, polygone
\item 3 propriétés des courbes B-Spline
\subsection{Algorithmes}
\item Algorithme d'évaluation. Cas d'une courbe de Bézier.
\item Algorithme de calcul de dérivées. 
\item Raccord entre deux courbes
\section{Approximation de surfaces}
\item Espace de Sobolev + norme
\item Injections canoniques, convergence faible
\item Injection canonique dans le cas des espaces de Sobolev
\item Injections compacts
\item Théorème de Stanpacchia
\item Théorème de Necas
\end{enumerate}

\part{Optimisation linéaire}
\begin{enumerate}
\item Définition infimum, minimum
\item Définition coercive
\item Deux exemples fonctions coercives
\item \textbf{Exemple de J coercive}
\item \textbf{Condition pour que J atteigne son minimum}
\item \textbf{Rapport frontière et minimum}
\item Définition dérivée directionnelle, Gâteaux-différentiable, gradient
\item Définition de Fréchet-différentiable
\item \textbf{Fréchet $\Rightarrow$ Gâteaux}
\item Définition espace convexe, épigraphe.
\item \textbf{Équivalence fonction convexe}
\item Définition strictement convexe, $\alpha$-convexe.
\item Équivalence à $f$ convexe
\item \textbf{Équivalence à $\alpha$-convexe}
\item \textbf{Condition d'optimalité dans un ouvert}
\item \textbf{Condition nécessaire puis condition suffisante pour un minimum}
\item \textbf{Théorème de Kuhn et Tucker}
\item Contraintes qualifiées
\item \textbf{Théorème dans le cas des contraintes qualifiées}
\item \textbf{Condition nécessaire de qualification}
\item \textbf{Lemme : équivalence à un minimum}
\item \textbf{Équivalent à minimum avec $\lambda$}
\item Point selle
\item \textbf{Propriété des points selles : système vérifié}
\item \textbf{Lemme : inégalité sup et inf}
\item \textbf{Problème dual}
\item Problème d'optimisation linéaire, passage du problème sous forme canonique à la forme standard
\item Ensemble des solutions réalisables, sommets
\item Définition de $\Gamma$, $A_{\gamma}$ et $\mathcal{B}$, composantes de/hors base. 
\item Définition de $x_X$, $x_N$, B, N. Redéfinition de $Ax$. 
\item Définition d'une solution de base associé à la base $\gamma$
\item Définition d'une solution de base réalisable, base réalisable. Solution non dégénéré.
\item \textbf{Les sommets de $X_{ad}$ sont exactement les solutions de base réalisable}
\item \textbf{S'il existe une solution optimale de $(P_L)$ alors il existe une solution optimale de base réalisable}
\item \textbf{Lemme sur $c^Tx$ et vecteur des prix marginaux}
\item \textbf{Lien vecteur des prix marginaux et solution optimale}
\item Définition de $E_{\gamma}$ et de $S_{\gamma, j^*}$
\item \textbf{Si $S_{\gamma, j^*}$ non vide ?}
\item \textbf{Redéfinition de $\delta$, comme quoi il appartient à B}
\item \textbf{Critère de Dantzig}
\item Définition des vraibles entrantes et sortantes selon le critère naturel et le critère de Bland
\item Problème de première ou deuxième espèce
\item \textbf{Comment construire une base réalisable dans le cas d'un problème de première espèce}
\item Bien connaître les problèmes de deuxième espèce
\item Définition du problème dual (et comment on y arrive)
\item \textbf{Inégalité entre primal et dual}
\item Corollaire
\item \textbf{Équivalence des solutions}
\item Définition de base primale ou duale réalisable.
\end{enumerate}

\part{Processus de Markov}
\begin{enumerate}
\item Définition de processus
\item Propriété de Markov, homogénéité
\item Mesure de probabilité, $f:E\to\mathbb{C}$ : représentation vectorielle
\item Matrice stochastique
\item \textbf{Relations de Kolmogorov}
\item Définition "$i$ conduit à $j$", conduit ) un préordre. Notation.
\item $i$ et $j$ communiquent, relation d'équivalence, notation.
\item Définition transitive, finale, ergodiques.
\item Existence de classes finales dans le cas fini
\item Forme canonique matrice de transition, puissance n
\item \textbf{Si E fini, alors le processus finira presque surement dans une des classes finales}
\item \textbf{A quoi correspond $(I-Q)^{-1}$ ?}
\item \textbf{B=NR ?}
\item Ensemble des entiers avec un chemin
\item Propriété fondamentale
\item \textbf{PGCD($N_{ii}$)}
\item Période d'une classe, classe apériodique.
\item \textbf{Forme de $N_{ii}$ et de $N_{ij}$}
\item Définition de $i\sim j$
\item Nombre de sous-classes cycliques
\item Chaîne régulière
\item \textbf{Équilvalence à chaîne régulière}
\item \textbf{Théorème fondamental des chaînes régulières}
\item \textbf{Théorème ergodique}
\item Générateur infinitésimal
\item Propriété des $a_j^i$
\item Théorème Backward et Forward
\item Définition d'absorbant, équivalence avec le générateur infinitésimal
\item Définition des instants de transition
\item Théorème fondamental
\item Chaîne discrète associée, classification, écriture du générateur infinitésimal
\item Limite $e^{t\chi}$ ?
\item Temps moyen passé par j sachant qu'on est parti de i, transitoires
\item Autre moyen de l'avoir
\item Probabilité de finir dans une classe finale donnée
\item Équivalence à $i\rightsquigarrow j$
\item Chaîne régulière en temps continu. Équivalence
\item Définition de loi de probabilité invariante. Équivalence avec générateur infinitésimal.
\item Théorème fondamental pour les chaînes régulières
\item Théorème ergodique en temps continu
\item Processus d'entrée-Sortie : définition de S et T
\item Lois de S et de T
\item Loi de l'instant de première transition
\item Savoir retrouver le générateur infinitésimal
\item Processus de Poisson, processus de comptage
\item Hypothèses des processus de comptage
\item Théorème sur la loi de ces processus
\item Théorème sur la loi des processus de Poisson
\item Loi de la durée entre deux arrivées
\item Truc sur un espace mesurable
\end{enumerate}

\part{Statistiques}
\begin{enumerate}
\section{Série chronologique}
\item Définition suite chronologique
\item Deux modèles de décomposition : quelle combinaison ? Dans quels cas va-t-on choisir l'un ou l'autre ?
\item Moindres carrés : expression de $a$ et $b$. 
\item Mise en place de la méthode des deux points
\item Expression de $r$
\item Moindre carré polynomial : expression de $\theta^{MC}$
\item Formule des moyennes mobiles
\item Définition MMC
\item Étape dans la décomposition d'une série chronologique
\section{Modèle Linéaire Gaussien}
\item Loi du chi2 : espérance et variance
\item Loi de Student, de Fisher, carré d'une Student
\item Définition MLG
\item Définition des estimateurs A, B et $\sigma^2$. Lois de chacun.
\item $d_x^2$ ?
\item Statistiques pour avoir les intervalles de confiance pour $\alpha$, $\beta$ et $\sigma^2$
\item IC de chacun d'entre eux
\item Test significatif du lien linéaire : Stat et zone de rejet
\item Test d'un modèle linéaire spécifique
\item IC pour $\mathbb{E}(Y_0)$ et pour une observation $Y_0$
\item Test du caractère significatif de la liaison linéaire par comparaison de modèle
\item ANOVA 1 : Gueule des données et du modèle
\item Definition dimension
\item Définition de $M_1$ et $M_p$
\item Estimation des paramètres dans chacun des modèles
\item Statistique de test
\item Définition contraste
\item Statistique de test pour les contrastes
\item Estimateur des $\mu_i$ et des $\sigma_i^2$
\item IC pour $\mu_i$
\section{ANOVA 1}
\item Facteur, niveau
\item Modèle de l'ANOVA 1 : modèle avec $\mu_i$ et avec les $\alpha_i$ (sans oublier les hypothèses)
\item Dimension du modèle
\item Approche comparaison de modèle : deux hypothèses testées. 5 étapes d'estimation dans modèle complet ou non
\item Statistique de test, sa loi sous $H_0$
\item Définition d'un contraste. 
\item Pour un contraste donné : hypothèses de test, statistique de test et loi
\item Estimations des paramètres $\mu_i$ : intervalle de confiance
\item Comment pourrait-on contruire un IC pour les $\alpha_i$ ?
\section{MLG multiple}
\item Variance d'un vecteur aléatoire. Propriétés de l'espérance et de la variance.
\item Écriture vectorielle du MLG multiple
\item Estimateur de $\theta$. Loi de T, d'une fonction de $S^2$
\item Construction d'un intervalle de confiance pour les $\theta_k$.
\item Lien entre loi de Student et combinaison linéaire de composantes de $\theta$
\item Revoir construction d'un IC pour une valeur inconnue $y_0$
\section{ANOVA 2}
\item Plan complet équilibré => Orthogonal
\item Modèle final d'ANOVA 2. Contrainte d'identifiabilité
\end{enumerate}
\end{document}
