\documentclass{article}
\input{../preambule}
%\usepackage[8pt]{extsizes}

\hypersetup{colorlinks=true, urlcolor=bleu, linkcolor=red}

%Def = Definition
%Theo = Théorème
%Prop = Propriété
%Coro = Corollaire
%Lem = Lemme

\makeatletter
\@addtoreset{section}{part}
\makeatother

\begin{document}

\setcounter{tocdepth}{4}
\tableofcontents
\newpage

\part{Automatique}
\begin{enumerate}
\item Définition linéaire, invariant.
\item Définition accessible, ensemble d'accessibilité, contrôlable, complètement contrôlable.
\item Solution de l'équation $\dot{x}(t)=Ax(t)+Bu(t)$
\item Définition transformée de Laplace. Transformée de $f'$. Transformée de cos, sin et exponentielle.
\item Fonction de transfert
\end{enumerate}

\part{\'El\'ements finis}
\begin{enumerate}
\item Théorème de Lax-Milgram
\item Choix de $V_h$, dimension ?
\item Fonctions de base, problème discrétisé
\item Matrice de rigidité
\item \textbf{R est définie positive}
\end{enumerate}

\part{Optimisation linéaire}
\begin{enumerate}
\item Définition infimum, minimum
\item Définition coercive
\item Deux exemples fonctions coercives
\item \textbf{Exemple de J coercive}
\item \textbf{Condition pour que J atteigne son minimum}
\item \textbf{Rapport frontière et minimum}
\item Définition dérivée directionnelle, Gâteaux-différentiable, gradient
\item Définition de Fréchet-différentiable
\item \textbf{Fréchet $\Rightarrow$ Gâteaux}
\item Définition espace convexe, épigraphe.
\item \textbf{Équivalence fonction convexe}
\item Définition strictement convexe, $\alpha$-convexe.
\item Équivalence à $f$ convexe
\item \textbf{Équivalence à $\alpha$-convexe}
\end{enumerate}

\part{Processus de Markov}
\begin{enumerate}
\item Définition de processus
\item Propriété de Markov, homogénéité
\item Mesure de probabilité, $f:E\to\mathbb{C}$ : représentation vectorielle
\item Matrice stochastique
\item \textbf{Relations de Kolmogorov}
\item Définition "$i$ conduit à $j$", conduit ) un préordre. Notation.
\item $i$ et $j$ communiquent, relation d'équivalence, notation.
\item Définition transitive, finale, ergodiques.
\item Existence de classes finales dans le cas fini
\item Forme canonique matrice de transition, puissance n
\item \textbf{Si E fini, alors le processus finira presque surement dans une des classes finales}
\item \textbf{A quoi correspond $(I-Q)^{-1}$ ?}
\item \textbf{B=NR ?}
\item Ensemble des entiers avec un chemin
\item Propriété fondamentale
\item \textbf{PGCD($N_{ii}$)}
\item Période d'une classe, classe apériodique.
\item \textbf{Forme de $N_{ii}$ et de $N_{ij}$}
\item Définition de $i\sim j$
\item Nombre de sous-classes cycliques
\item Chaîne régulière
\item \textbf{Équilvalence à chaîne régulière}
\item \textbf{Théorème fondamental des chaînes régulières}
\item \textbf{Théorème ergodique}
\end{enumerate}

\part{Statistiques}
\begin{enumerate}
\item Définition suite chronologique
\item Deux modèles de décomposition : quelle combinaison ? Dans quels cas va-t-on choisir l'un ou l'autre ?
\item Moindres carrés : expression de $a$ et $b$. 
\item Mise en place de la méthode des deux points
\item Expression de $r$
\item Moindre carré polynomial : expression de $\theta^{MC}$
\end{enumerate}
\end{document}
