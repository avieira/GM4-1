\part{Conditions d'optimalit\'e}

On considère le problème d'optimisation suivant :
\[\inf_{u\in\mathcal{U}_{ab}} J(u) \ (\mathcal{P}_{\mathcal{U}_{ab}})\]

Ici, $J:\mathbb{R}^d\to \mathbb{R}$ et $\mathcal{U}_{ab}$ est un sous-ensemble non vide de $\mathbb{R}^d$. \\
J s'appelle la fonction coût.\\
$\mathcal{U}_{ab}$ s'appelle l'ensemble admissible.

\section{Existence d'un minimum}
\Def{}{Soit $J : \mathbb{R}^d\to \mathbb{R}$ et $\mathcal{U}_{ad}$ un sous-ensemble non vide.\\
On dit que $l\in[-\infty,+\infty[$ est l'infimum de $J$ sur $\mathcal{U}_{ad}$ si :
\begin{enumerate}
\item $J(u)\geq l\ \forall u\in\mathcal{U}_{ad}$
\item $\exists (u_n)_n\subset \mathbb{R}^d;\ u_n\in\mathcal{U}_{ad}$ et $J(u_n)\to l$
\end{enumerate}

On note $l=\inf_{u\in\mathcal{U}_{ad}} J(u)$ et les suites vérifiant $(u_n)_n \subset \mathcal{U}_{ad}$ et $J(u_n)\xrightarrow[n\to+\infty]{} \inf_{\mathcal{U}_{ad}} J(u)$ sont appelées suites minimisantes.}

\textit{\underline{Remarque :}} L'infimum existe toujours. Il est fini si et seulement si J est minorée.

\Def{}{Soit $J : \mathbb{R}^d\to \mathbb{R}$ et $\mathcal{U}_{ab}$ un sous-ensemble non vide.\\
On dit que $l\in\mathbb{R}$ est le minimum de J sur $\mathcal{U}_{ad}$ (si cette valeur existe) si on a :
\begin{enumerate}
	\item $J(u)\geq l\ \forall u\in\mathcal{U}_{ad}$
	\item $\exists \bar{u}\in\mathcal{U}_{ad}$ tel que $J(\bar{u})=l$
\end{enumerate}
On dit alors que $J$ atteint son minimum sur $\mathcal{U}_{ad}$ et on note $l=\min_{u\in\mathcal{U}_{ad}} J(u)$}

\textit{\underline{Remarque :}} \begin{enumerate}
\item Le minimum n'existe pas toujours
\item Par abus de langage, on appelle aussi minimum le point $\bar{u}$ qui vérifie $J(u)\geq J(\bar{u})\ \forall u\in\mathcal{U}_{ad}$ ($\bar{u}$ est l'argument du minimum).
\end{enumerate}

\Def{}{On dit que $J:\mathbb{R}^d\to\mathbb{R}$ est coercive si : \[\lim_{\|u\|\to+\infty} = +\infty\]}

\underline{\textit{Remarque :}} En dimension finie, toutes les normes sont équivalentes. Il suffit donc de le vérifier pour la norme la plus "facile"

\Exemp{}{\begin{enumerate}
		\item Soit A une matrice symétrique de taille d, $b\in\mathbb{R}^d$ et $c$ un réel.\\
On considère l'application :
\begin{eqnarray*}
J : \mathbb{R}^d &\to& \mathbb{R}\\
	u &\mapsto& \langle Au,u\rangle + \langle b,u\rangle + c
\end{eqnarray*}
$J$ est coercive si et seulement si A est définie positive.

Si A est symétrique, on a :
\[\lambda_{min}\|u\|^2 \leq \langle Au,u\rangle \leq \lambda_{max}\|u\|^2\]
où $\lambda_{min}$ est la plus petite valeur propre, et $\lambda_{max}$ est la plus grande valeur propre.

\item Toute fonction minorée par une fonction coercive est coercive.
	\end{enumerate}
}

\Prop{}{On suppose que $J:\mathbb{R}^d\to\mathbb{R}$ prend la forme :
	\[\forall u=(u_1,...,u_d)\in\mathbb{R}^d, J(u)=\sum_i J_i(u_i)\]
avec $J_i:\mathbb{R}\to\mathbb{R}$ coercive et minorée. Alors J est coercive.}

\begin{dem}
$\forall i\in\{1,...,d\}$, $J_i$ est minorée par une constante $m_i$. \\
On note $m=\max_{i\in\{1,...,d\}} |m_i|$. 

\bigskip
Soit $M>0$ fixé. $\forall i\in\{1,...,d\}$, $\exists R_i>0;\ \forall |u_i|>R_i$, on a $J_i(u_i)>M+md$ (car $J_i$ coercive).\\
Posons $R=\max_{i\in\{1,...,d\}} R_i$

\bigskip
Soit $u\in\mathbb{R}^d$; $\|u\|_{\infty}\geq R$. 
\[\Rightarrow \exists j\in\{1,...,d\};\ |u_j|\geq R\geq R_j\]
On a donc $J_j(u_j)\geq M+md$\\
Ainsi :
\begin{eqnarray*}
	J(u)&=&\sum_i J_i(u_i)\\
	    &=&J_j(u_j)+\sum_{i\neq j} J_i(u_i)\\
	    &\geq& M+md + \sum_{j\neq j} m_i \\
	    &\geq& M+\sum_{i\neq j} (m+m_i)\\
	    &\geq& M
\end{eqnarray*}
On a montré que :
\[\forall M>0, \exists R; \forall u;\ \|u\|_{\infty}\geq R \Rightarrow J(u)\geq M\]
ie $\lim_{\|u\|_{\infty}\to\infty} J(u)=+\infty$ \\
Par conséquent, J est coercive.
\end{dem}

\Theo{}{Soit $J:\mathbb{R}^d\to\mathbb{R}$ une fonction continue et $\mathcal{U}_{ad}$ un ensemble fermé non vide. On suppose que :
\begin{itemize}
	\item Soit J est coercive
	\item Soit $\mathcal{U}_{ad}$ est borné
\end{itemize}
Alors J atteint son minimum sur $\mathcal{U}_{ad}$.}

\Prop{}{Soit $\mathcal{U}_{ad}$ un ouvert fermé de $\mathbb{R}^d$ et soit $J:\mathbb{R}^d\to\mathbb{R}$ une fonction continue.\\
On suppose qu'il existe $u_0\in\mathcal{U}_{ad}$ tel que :
\[J(u_0)<J(u)\ \forall u\in\partial \mathcal{U}_{ad}\]
où $\partial \mathcal{U}_{ad}$ est la frontière de $\mathcal{U}_{ad}$.\\
Alors J atteint son minimum sur $\mathcal{U}_{ad}$.}

\begin{dem}
$\overline{\mathcal{U}_{ad}}$ est un compact et $J$ est continue, donc :
\[\exists \bar{u}\in\overline{\mathcal{U}_{ad}};\ J(\bar{u})\leq J(u)\forall u\in\overline{\mathcal{U}_{ad}}\]
Montrons que $\bar{u}\in\mathcal{U}_{ad}$. \\
Si $\bar{u}\not\in\mathcal{U}_{ad}$, alors $\bar{u}\in\partial \mathcal{U}_{ad}$ et par hypothèse, on a $J(\bar{u})>J(u_0)$ avec $u_0\in\mathcal{U}_{ad}$. Contradiction.\\
Donc $\bar{u}\in\mathcal{U}_{ad}$ et $J(\bar{u})\leq J(u) \forall u\in\mathcal{U}_{ad}$. 
\end{dem}

\section{Conditions nécessaires d'optimalité}
\subsection{Plusieurs notions de dérivabilité}

\Def{Dérivées directionnelle}{Soit $H_1$ et $H_2$ deux espaces de Hilbert et $f:H_1\to H_2$.\\
On appelle dérivée directionnelle de f au point $x\in H_1$ dans la direction $d\in H_1$, notée $f'(x,d)$, la limite (si elle existe) : 
\[f'(x,d)=\lim_{\varepsilon\to 0^+} \frac{f(x+\varepsilon d)-f(x)}{\varepsilon}\]}

\Def{Gâteaux-différentiabilité}{On dit que $f$ est Gâteaux-différentiable en $x\in H_1$ si $f$ admet des dérivées directionnelles au point $x$ dans toutes les directions et si l'application
	\[d\in H_1 \mapsto f'(x,d)\]
est linéaire continue.

\bigskip
On note alors $f'(x)$ cette application :
\[f'(x,d)=f'(x) d\ \forall d\in H_1\]
On dit que $f$ est Gâteaux-différentiable en tout point $x\in H_1$.}

\Def{}{Soit $f: H\to \mathbb{R}$ une fonction Gâteaux-différentiable en $x\in H$.\\
On note $\nabla f(x)$ (appelé gradient de $f$ au point $x$) l'unique élément de $H$ tel que 
\[f'(x) d = \langle \nabla f(x),d\rangle\]}

\Def{}{On dit que $f:H_1 \to H_2$ est Fréchet-différentiable en $x$ s'il existe une application linéaire continue de $H_1$ dans $H_2$ tel que :
	\[\lim_{h\to 0} \frac{f(x+h)-f(x)-Lh}{\|h\|}=0\]
L'opérateur $L$ est appelé la dérivée de f en a.}

\Prop{}{Soient $H_1$ et $H_2$ deux espaces de Hilbert et $f:H_1\to H_2$. On suppose que $f$ est Fréchet-différentiable en $x\in H_1$ avec une dérivée L. Alors $f$ est Gâteaux-différentiable et $L=f'(x)$.}

\begin{dem}
Soit $d\neq0\in H_1$. $\forall h>0$ :
\[\begin{array}{c c c c}
 &\frac{f(x+hd)-f(x)-Lhd}{h\|d\|}&\rightarrow&0 \\
\Rightarrow &\frac{f(x+h)-f(x)}{h}-Ld&\rightarrow& 0 \\
\Rightarrow & \frac{f(x+h)-f(x)}{h}&\rightarrow& Ld
\end{array}\]

$f$ admet une dérivée directionnelle dans la direction $d$, et on a $f'(x,d)=Ld$, ie $f'(x,\bullet)=L$ est linéaire continue.
\end{dem}

\subsection{Quelques rappels d'analyse convexe}
\Def{}{Soit C un sous ensemble d'un espace vectoriel. On dit que C est convexe si \[\forall x,y\in C, \forall \alpha\in[0,1], \alpha x + (1-\alpha)y\in C\]}

\Def{}{Soit $f:H\to \mathbb{R}$. On appelle épigraphe de $f$, noté $\text{epi}(f)$, l'ensemble :
\[\text{epi}(f)=\{(\alpha,x)\in\mathbb{R}\times H; \alpha\geq f(x)\}\]}

\Def{}{Soit $f:H\to \mathbb{R}$. On dit que $f$ est convexe si epi$(f)$ est convexe.}

\Prop{}{$f:H\to \mathbb{R}$ est convexe si et seulement si $\forall x,y\in H$, $\forall \alpha\in[0,1]$, on a : \[f(\alpha x+(1-\alpha)y)\leq \alpha f(x)+(1-\alpha)f(y)\]}

\begin{dem}
On suppose que $f$ est convexe. Soient $x,y\in H$ et $\alpha\in[0,1]$. \\
On a $(f(x),x)$ et $(f(y),y)\in\text{epi}(f)$. Donc $\alpha(f(x),x)+(1-\alpha)(f(y),y)\in\text{epi}(f)$. 
\[\Rightarrow f(\alpha x+(1-\alpha)y)\leq \alpha f(x)+(1-\alpha)f(y)\]

\bigskip
Réciproquement, on suppose $\forall x,y\in H$, $\forall \alpha\in[0,1]$, \[f(\alpha x+(1-\alpha)y)\leq \alpha f(x) + (1-\alpha)f(y)\]
Soient $(\alpha,x)$ et $(\beta,y)\in\text{epi}(f)$ et $\lambda\in[0,1]$. \[\alpha\geq f(x) \text{ et } \beta\geq f(y)\]
On a $f(\lambda x + (1-\lambda)y)\leq \lambda f(x)+(1-\lambda)f(y) \leq \lambda \alpha + (1-\lambda)\beta$\\
\[\Rightarrow (\lambda\alpha+(1-\lambda)\beta, \lambda x+(1-\lambda)y)\in\text{epi}(f)\]
Donc epi$(f)$ est convexe.
\end{dem}

\Def{}{Soit $f:H\to \mathbb{R}$. On dit que $f$ est strictement convexe si $\forall x,y\in H$, tel que $x\neq y$, $\forall \alpha\in[0,1]$, on a :
\[f(\alpha x+(1-\alpha)y)<\alpha f(x)+(1-\alpha)f(y)\]}

\Def{}{Soit $f:H\to\mathbb{R}$. On dit que $f$ est $\alpha$-convexe si $\forall x,y\in H,\ \forall \lambda\in[0,1]$, on a :
\[\frac{\alpha}{2}\lambda(1-\lambda)\|x-y\|^2+f(\lambda x+(1-\lambda)y)\leq \lambda f(x)+(1-\lambda)f(y)\]}

\Prop{Convexité et dérivée première}{Soit $f:H\to \mathbb{R}$ différentiable. On a équivalence entre les propositions suivantes :
\begin{enumerate}
\item $f$ est convexe
\item $f(y)\geq f(x)+f'(x,y-x)$
\item $(f'(y)-f'(x))(y-x)\geq 0$
\end{enumerate}}

\Prop{$\alpha$-convexité  et dérivée première}{Soit $f:H\to \mathbb{R}$ différentiable. On a équivalence entre les propositions suivantes :
\begin{enumerate}
	\item $f$ est $\alpha$-convexe
	\item $f(y)\geq f(x)+f'(x).(y-x)+\frac{\alpha}{2}\|x-y\|^2$
	\item $(f'(y)-f'(x)).(y-x)\geq \alpha\|x-y\|^2$
\end{enumerate}}

\begin{dem}
$(1)\to(2)$\\
On suppose que $f$ est $\alpha$-convexe. Par définition, on a, pour $\lambda=\frac{1}{2^k}$ : 
\[f\left(\left(1-\frac{1}{2^k}\right)x+\frac{1}{2^k}y\right)\leq \left(1-\frac{1}{2^k}\right) f(x)+\frac{1}{2^k}f(y)-\frac{\alpha}{2^{k+1}}\left(1-\frac{1}{2^k}\right)\|x-y\|^2\]
On a alors :
\[2^k\left[f\left(\left(1-\frac{1}{2^k}\right)x+\frac{1}{2^k}y\right)-f(x)\right]\leq f(y)-f(x)-\frac{\alpha}{2}\left(1-\frac{1}{2^k}\right)\|x-y\|^2\]
Lorsque $k\to +\infty$ : 
\[f'(x).(y-x)\leq f(y)-f(x)-\frac{\alpha}{2}\|x-y\|^2\]

\bigskip
$(2)\to (3)$\\
\[f(y)\geq f(x)+f'(x)(y-x)-\frac{\alpha}{2}\|x-y\|^2\]
\[f(x)\geq f(y)+f'(y)(x-y)-\frac{\alpha}{2}\|x-y\|^2\]
On fait la somme :
\[0\geq (-f'(x)+f'(y)).(x-y)+\alpha\|x-y\|^2\]

\bigskip
$(3)\to(1)$\\
Soient $x,y\in H$ et $\lambda\in[0,1]$. On introduit :
\begin{eqnarray*}
\phi : \mathbb{R}&\to&\mathbb{R}\\
	t&\mapsto&f(x+t(y-x))
\end{eqnarray*}

$\phi$ est dérivable et \[\phi'(t)=f'(x+t(y-x)).(y-x)\]
Soit $t>s$. 
\begin{eqnarray*}
\phi'(t)-\phi'(s)&=&[f'(x+t(y-x))-f'(x+s(y-x))].(y-x) \\
	&\geq& \frac{1}{t-s}\alpha\|(t-s)(x-y)\|^2\\
	&\geq& \alpha(t-s)\|x-y\|^2
\end{eqnarray*}
On intègre de $t=\lambda$ à $t=1$ et de $s=0$ à $s=\lambda$ :
\[\begin{array}{c c c c}
 & \lambda(\phi(1)-\phi(\lambda))-(1-\lambda)(\phi(\lambda)-\phi(0)) &\geq& \alpha\|y-x\|^2\left[ \frac{\lambda}{2} (1-\lambda^2) - \frac{\lambda^2}{2}(1-\lambda)\right]\\
\Leftrightarrow & \lambda\phi(1)+(1-\lambda)\phi(0)-\phi(\lambda) &\geq& \alpha\|x-y\|^2\frac{\lambda}{2}(1-\lambda^2-\lambda+\lambda^2)\\
\Leftrightarrow & \lambda f(y) + (1-\lambda)f(x)&\geq& \frac{\alpha}{2}\|x-y\|^2\lambda(1-\lambda)+f((1-\lambda)x+\lambda y)
\end{array}\]
Donc f $\alpha$-convexe.
\end{dem}
