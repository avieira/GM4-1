\subsection{Conditions d'optimalité dans un ouvert}
\Theo{}{Soit $J:\mathbb{R}^n\to\mathbb{R}$ une fonction de classe $\mathcal{C}^1$. Soit $\mathcal{U}_{ad}$ un ouvert de $\mathbb{R}^n$.\\
Si $J$ atteint un minimum local en $\bar{u}\in\mathcal{U}_{ad}$, alors \[\nabla J(\bar{u})=0\]}

\underline{Remarque :} Il existe également une condition du second ordre (si $J$ est $\mathcal{C}^2$) : la matrice Hessienne est positive.

\begin{dem}
Soit $d\in\mathbb{R}^n$. Comme $\mathcal{U}_{ad}$ est ouvert et $\bar{u}\in\mathcal{U}_{ad}$, il existe $h_0>0;\ \forall h\in]0,h_0],\ \bar{u}+hd\in\mathcal{U}_{ad}$. Donc $J(\bar{u}+hd)\geq J(\bar{u})$.

Or :
\begin{eqnarray*}
& & J(\bar{u})\leq J(\bar{u}+hd)=J(\bar{u})+\langle \nabla J(\bar{u}),hd\rangle + o(h) \\
&\Rightarrow& \langle\nabla J(\bar{u}), hd\rangle +o(h)\geq 0 \\
&\Rightarrow& \langle \nabla J(\bar{u}),h\rangle + o(1)\geq 0 \\
&\Rightarrow& \langle J(\bar{u}),d\rangle \geq 0\ \forall d\in\mathbb{R}^n\ (h\to 0)
\end{eqnarray*}

En remplaçant $d$ par $-d$ :
\[\langle\nabla J(\bar{u}),d\rangle\leq 0\ \forall d\in\mathbb{R}^n\]

D'où :
\[\langle \nabla J(\bar{u}),d\rangle =0\ \forall d\in\mathbb{R}^n\]
\[\Rightarrow \nabla J(\bar{u})=0\] 
\end{dem}

\Theo{}{Soit $\mathcal{U}_{ad}$ un ensemble convexe de $\mathbb{R}^n$ et $J$ une application différentiable de $\mathbb{R}^n$ dans $\mathbb{R}$. Si $\bar{u}$ est un point de minimum de $J$ sur $\mathcal{U}_{ad}$, alors :
\begin{equation} \tag{$*$} J'(\bar{u})(u-\bar{u})\geq 0\ \forall u\in\mathcal{U} \end{equation}
Réciproquement, si $\bar{u}$ vérifie $(*)$, et si $J$ convexe, alors $\bar{u}$ est un point de minimum de $J$.} 

\begin{dem}
Soit $u\in\mathcal{U}_{ad}$. On a $\forall h\in[0,1]$, $\bar{u}+h(u-\bar{u})=(1-h)\bar{u}+hu\in\mathcal{U}_{ad}$ (par convexité de $\mathcal{U}_{ad}$). \\
Donc \[J(\bar{u}+h(u-\bar{u}))\geq J(\bar{u})\]
\[\Rightarrow \forall u\in\mathcal{U}_{ad},\ \forall h\in[0,1],\ \frac{J(\bar{u}+h(u-\bar{u}))-J(\bar{u})}{h}\geq 0\]
\[\Rightarrow J'(\bar{u})(u-\bar{u})\geq 0\ (h\to 0^+)\]

\bigskip
Réciproquement, si $J$ est convexe, et si $\bar{u}$ vérifie $(*)$, alors $\forall u\in\mathcal{U}_{ad}$ :
\[0\leq J'(\bar{u})(u-\bar{u})\leq J(u)_J(\bar{u})\]
car J convexe, d'où :
\[J(u)\geq J(\bar{u})\ \forall u\in\mathcal{U}_{ad}\]
\end{dem}

\Rem{}{On considère le cas où $\mathcal{U}_{ad}$ est un sous-espace affine de $\mathbb{R}^n$. En particulier, :
\[\mathcal{U}_{ad}=\mathcal{P}+\bar{u}\]
où $\mathcal{P}$ est un espace vectoriel.\\
La condition $(*)$ se réécrit :
\[J'(u)v\geq 0\ \forall v\in\mathcal{P}\]
Si $v\in\mathcal{P}$ alors $-v\in\mathcal{P}$, donc :
\[J'(u).v=0,\ \forall v\in\mathcal{P}\]
ie $\nabla J(\bar{u})\in\mathcal{P}^\perp$

\bigskip
En particulier, si $\mathcal{P}$ est défini comme une intersection (finie) d'hyperplan ($a_i\in\mathbb{R}^n$) :
\[\mathcal{P}=\{x,\ \langle a_i,x\rangle =0\ \forall i=1,...,d\}\]
alors $\mathcal{P}^\perp$ est engendré par la famille $(a_i)_{i=1...d}$
La condition d'optimalité s'écrit :
\[\exists (\lambda_i)_{i=1,...,d};\ \nabla J(\bar{u})+\sum_{i=1}^d \lambda_ia_i=0\]
Les $\lambda_i$ sont appelés multiplicateurs de Lagrange.}

\subsection{Théorème de Kuhn et Tucker}
On suppose que la contrainte $\mathcal{U}_{ad}$ s'écrit :
\[\mathcal{U}_{ad}=\left\{u\in\mathbb{R}^n;\ \begin{array}{c c c c} 
g_i(u)&\leq&0 & \forall i\in I \\
h_j(u)&=& 0 & \forall j\in J
\end{array}\right\}\]
om $I=\{1,...,l\}$ et $J=\{1,...,m\}$. \\
On suppose que les fonctions $g_i$ et $h_j$ sont $\mathcal{C}^1$ et pour $u\in\mathcal{U}_{ad}$, on note $I(u)$ l'ensemble des contraintes saturées, ie :
\[I(u)=\{i\in I;\ g_i(u)=0\}\]

\Theo{}{Si $\bar{u}$ est un point de minimum local de $J$ sur $\mathcal{U}_{ad}$ alors il exsite $p_0\in\mathbb{R}^+$, $p\in\mathbb{R}_+^l$, $q\in\mathbb{R}^m$, tel que :
\[\left\{\begin{array}{c c c c}
\sum_{i\in I} p_ig_i(\bar{u})&=&0 & \text{ (condition d'exclusion)} \\
(p_0,p,q)&\neq& 0 \\
p_0\nabla J(\bar{u})+\sum_{i\in I} p_0 \nabla g_i(\bar{u}) + \sum_{j\in J} q_j \nabla h_j(\bar{u})&=&0 & \text{ (condition nécessaire)}
\end{array}\right.\]}

\begin{dem}
Soit $r>0$ tel que $J$ atteigne un minimum sur $\overline{B(\bar{u},r)}$ en $\bar{u}$.

\[\min_{u\in\overline{B(\bar{u},r)}} \left\{J(u)+\|u-\bar{u}\|^2+\frac{N}{2}\left(\sum_{i\in I} \max (g_i(u),0)^2 + \sum_{j\in J} \max(h_j(u),0)^2 \right) \right\}\]
Le minimum est atteint en $\bar{u}_N$.

\bigskip
Comme $J$ est continue sur $\overline{B(\bar{u},r)}$, elle est bornée. On note $M=\|J\|_{L^\infty(\overline{B(\bar{u},r)})}$. 

\bigskip
Donc :
\[\sum_{i\in I} \max (g_i(u),0)^2 + \sum_{j\in J} \max(h_j(u),0)^2 \leq \frac{2}{N} \left(J(\bar{u})-J(\bar{u}_n)-\|\bar{u}-\bar{u}_N\|^2\right)\]
\begin{equation}\tag{$*$}
\sum_{i\in I} \max (g_i(\bar{u}_N),0)^2 + \sum_{j\in J} \max(h_j(\bar{u}_N),0)^2 \leq \frac{2}{N}(2M+r)
\end{equation}

On a aussi :
\begin{equation}\tag{$**$}
J(\bar{u}_N)+\|\bar{u}-\bar{u}_N\|^2 \leq J(\bar{u})
\end{equation}

Comme $\overline{B(\bar{u},r)}$ est compacte, quitte à extraire une sous-suite, on peut supposer que $\bar{u}_N\to u^*\in\overline{B(\bar{u},r)}$.\\
En prenant la limite dans $(*)$ et dans $(**)$.
\[\sum_{i\in I} \max (g_i(u^*),0)^2 + \sum_{j\in J} \max(h_j(u^*),0)^2 =0\]
et
\[J(u^*)+\|\bar{u}-u^*\|^2 \leq J(\bar{u})\]

Donc $\forall i$, $g_i(u^*)\leq 0$ et $\forall j$, $h_j(u^*)\leq 0$.
\[\Rightarrow u^*\in\mathcal{U}_{ad}\]
\[\Rightarrow u^*=\bar{u}\]

Pour N assez grand, $\bar{u}_N\in B(\bar{u},r)$. On en déduit donc que :
\begin{equation}
\tag{$*$}
\nabla J(\bar{u}_N)+2\|\bar{u}_N-\bar{u}\|+N\left(\sum_{i\in I} \max(g_i(\bar{u}_N),0)\nabla g_i(\bar{u}_N)+\sum_{j\in J} h_j(\bar{u}_N)\nabla h_j(\bar{u}_N)\right)=0
\end{equation}

On pose \[\rho_N=\left[ 1+N^2\sum_{i\in I} \max(0,g_i(\bar{u}_N))^2 + N^2\sum_{j\in J} h_j(\bar{u}_N)^2\right]^{\frac{1}{2}}\]

On pose :
\[p_0^N=\frac{1}{\rho_N}\]
\[p_i^N=Np_0^N \max(0,g_i(\bar{u}_N))\]
\[q_j^N=Np_0^N h_j(\bar{u}_N))\]

Le vecteur $\left( p_0^N, p^N, q^N\right)\in\mathbb{R}^{p+m+1}$ est de norme 1.

\bigskip
Quitte à extraire une sous-suite, on peut supposer que $(p_0^N, p^N, q^N)\to (p_0,p,q)$ avec $\|(p_0,p,q)\|=1$. 

\bigskip
En utilisant le fait que $\bar{u}_N\to\bar{u}$ et en divisant $(*)$ par $\rho_N$ puis en passant à la limite, on a :
\[p_0\nabla J(\bar{u})+\sum_{i\in I} p_i\nabla g_i(\bar{u})+\sum_{j\in J} q_j\nabla h_j(\bar{u})=0\]
Il reste à montrer que :
\[\sum_{i\in I} p_ig_i(\bar{u})=0\]

Or, $\forall i,\ p_ig_i(\bar{u})<0$. Il faut donc montrer que :
\[p_ig_i(\bar{u})=0 \forall i\]

ie : 
\[p_i=0 \forall i\not\in I(\bar{u})\]

Si $i\not\in I(\bar{u})<0$ alors $g_i(\bar{u})<0$. Donc $g_i(\bar{u}_N)<0$ pour N assez grand. Donc $p_i=0$ pour N assez grand.
\[\Rightarrow p_i=0\ \forall i\not\in I(\bar{u})\]
\[\Rightarrow \sum_{i\in I} p_ig_i(\bar{u})=0\]
\end{dem}

\underline{Remarque :} \begin{enumerate}
\item Le vecteur $(p_0,p,q)$ est appelé le multiplicateur de Lagrange généralisé associé à $\bar{u}$
\item On appelle lagrangien généralisé :
\[L(u,p_0,p,q)=p_0J(u)+\sum_{i\in I} p_ig_i(u)+\sum_{j\in J} q_jh_j(u)\]
La condition nécessaire d'optimalité se réécrit :
\[\nabla_u L(\bar{u},p_0,p,q)=0\]
\end{enumerate}

\subsection{Cas des contraintes qualifiées}
\Def{Contraintes qualifiées}{On dit que les contraintes sont qualifiées en un point $\bar{u}$ de $\mathcal{U}_{ad}$ si les conditions suivantes sont vérifiées :
\begin{enumerate}
	\item $\left\{ \nabla h_1(\bar{u}),...,\nabla h_n(\bar{u})\right\}$
	\item $\exists v\in \mathbb{R}^n$ tel que :
		\[\langle \nabla h_j(\bar{u}),v\rangle=0\ \forall j=1,...,n\]
	et	\[\langle \nabla g_i(\bar{u}),v\rangle<0\ \forall i\in I(\bar{u}) \]
\end{enumerate}}

\Theo{}{Si $\bar{u}$ est un point de minimum de $J$ sur $\mathcal{U}_{ad}$ et si les contraintes sont qualifiées en $\bar{u}$, alors $\exists \lambda\in\mathbb{R}^l_+$, $\mu\in\mathbb{R}^n$ tel que :
\begin{enumerate}
	\item \[\sum_{i\in I} \lambda_i g_i(\bar{u})=0 \text{ (condition d'exclusion)}\]
	\item \[\nabla J(\bar{u})+\sum_{i\in I} \lambda_i \nabla g_i(\bar{u})+\sum_{j\in J} \mu_j \nabla h_j(\bar{u})=0 \text{ (condition d'optimalité)}\]
\end{enumerate}}

\begin{dem}
D'après le théorème précédent, il existe $p_0\in\mathbb{R}_+$, $p\in\mathbb{R}^l_+$ et $q\in\mathbb{R}^m$ tel que :
\[\left\{\begin{array}{c c c }
\sum_{i\in I} p_ig_i(\bar{u})&=&0 \\
(p_0,p,q)&\neq& 0 \\
p_0\nabla J(\bar{u})+\sum_{i\in I} p_0 \nabla g_i(\bar{u}) + \sum_{j\in J} q_j \nabla h_j(\bar{u})&=&0 
\end{array}\right.\]
Montrons que $p_0\neq 0$.\\
Par l'absurde, on suppose $p_0=0$.
\begin{enumerate}
	\item $p_i=0 \forall i\not\in I(\bar{u})$
	\item $\sum_{i\in I(\bar{u})} p_i \nabla g_i(\bar{u}) + \sum_{j\in J} q_j \nabla h_j(\bar{u})=0$
\end{enumerate}
Soit $v\in\mathbb{R}^n$ tel que $\langle \nabla h_j(\bar{u}),v\rangle=0\ \forall j\in J$ et $\langle \nabla h_i(\bar{u}),v\rangle<0\ \forall i\in I(\bar{u})$.\\
On a :
\begin{eqnarray*}
0&=&\left\langle \sum_{i\in I(\bar{u})} p_i \nabla g_i(\bar{u}) + \sum_{j\in J} q_j \nabla h_j(\bar{u}),v\right\rangle \\
 &=&\sum_{i\in I(\bar{u})} \underbrace{p_i}_{\geq 0} \underbrace{\langle \nabla g_i(\bar{u}), v\rangle}_{<0} + \underbrace{\sum_{j\in J} q_j\langle \nabla h_j(\bar{u}),v\rangle}_{=0}
\end{eqnarray*}
\[\Rightarrow p_i=0\ \forall i\in I(\bar{u})\]
\[\Rightarrow p=0\]
Donc :
\[\sum_{j\in J} q_j \nabla h_j(\bar{u})=0\]
Or, $\{\nabla h_1(\bar{u}),...,\nabla h_m(\bar{u})\}$ forment une famille libre, donc on a forcément $\forall j\in J,\ q_j=0$, ie : \[q=0\]

On en déduit donc que \[(p_0,p,q)=0\] ce qui est absurde.\\
Donc $p_0\neq 0$.

\bigskip
On pose $\lambda_i=\frac{p_i}{p_0}>0$ et $\mu_i=\frac{q_i}{p_0}$. On retrouve ainsi les deux égalités.
\end{dem}

\Rem{}{\begin{enumerate}
	\item Le résultat reste vrai sans les hypothèses de qualification si les contraintes sont affines (ie, $\forall i, \forall j$, $g_i$ et $h_j$ sont convexes ou concaves)
	\item Un peu de vocabulaire :
	\begin{itemize}
		\item $(\lambda,\mu)$ est le multiplicateur de Lagrange associé à $\bar{u}$
		\item Le lagrangien est défini par :
			\[L(u,\lambda,\mu)=J(u)+\sum_{i\in I}\lambda_i g_i(u) + \sum_{j\in J} \mu_j h_j(u)\]
			et la condition d'optimalité s'écrit :
			\[\nabla_u L(\bar{u},\lambda,\mu)=0\]
	\end{itemize}
\end{enumerate}}

\section{Problèmes convexes et dualité}
On considère le problème :
	\[\min_{u\in \mathcal{U}_{ad}} J(u)\]
où $\mathcal{U}_{ad}=\{u\in\mathbb{R}^n;\ g_i(u)\leq 0 \forall i\in I\}$

\bigskip
On suppose que les applications $J,g_1,...,g_l$ sont convexes et de classe $\mathcal{C}^1$.

\Prop{}{On suppose qu'il existe $u_0\in\mathcal{U}_{ad}$ tel que \[g_i(u_0)<0\ \forall i\in I\]
Alors la contrainte est qualifiée (en $\bar{u}$, $\forall\bar{u}\in\mathcal{U}_{ad}$).}

\begin{dem}
Soit $u\in\mathcal{U}_{ad}$. On pose $v=u_0-u$. Soit $j\in I(\bar{u})$. Comme $g_i$ est convexe et de classe $\mathcal{C}^1$, on a :
	\[\langle \nabla g_i, v\rangle \leq \underbrace{g_i(u_0)}_{<0}-\underbrace{g_i(\bar{u})}_{=0}<0\]
La contrainte est donc qualifiée en $\bar{u}$.
\end{dem}

\Theo{}{$L(u,\lambda)=J(u)+\lambda^Tg(u)$\\
Si $\bar{u}$ est un minimum de $J$ sur $\mathcal{U}_{ad}$ et si les contraintes sont qualifiées en $\bar{u}$, alors $\exists \lambda\in\mathbb{R}^l_+$ tel que :
	\[\left\{ \begin{array}{c c c}
		\nabla_u L(\bar{u},\lambda)&=&0\\
		\lambda^Tg(\bar{u})=0
	\end{array}\right.\]}

\Lem{}{Soit $F$ une application de $\mathbb{R}^n$ dans $\mathbb{R}$ convexe et de classe $\mathcal{C}^1$. $u\in\mathbb{R}^n$ est un point de minimum de $F$ sur $\mathbb{R}^n$ si et seulement si $\nabla F(u)=0$.}

\begin{dem}
Une première implication a déjà été montrée.\\
On suppose que $\nabla F(u)=0$. Comme F est convexe, on a \[F(v)-F(u)\geq \langle \underbrace{\nabla F(u)}_{=0},v-u\rangle \forall\ v\in\mathbb{R}^n\]
D'où $F(v)\geq F(u)$ $\forall v\in\mathbb{R}^n$, donc $u$ est un minimum de F sur $\mathbb{R}^n$.
\end{dem}

\Theo{}{Soit $\bar{u}$ un point de $\mathcal{U}_{ad}$ tel que les contraintes soient qualifiées en $\bar{u}$. Alors $\bar{u}$ est un point de minimum de $J$ sur $\mathcal{U}_{ad}$ si et seulement si $\exists \lambda\in\mathbb{R}^l_+$ tel que : 
	\[\left\{ \begin{array}{c c c}
		\nabla_u L(\bar{u},\lambda)&=&0\\
		\lambda^Tg(\bar{u})=0
	\end{array}\right.\]}

\begin{dem}
L'application $u\mapsto L(u,\lambda)$ est une fonction convexe (en tant que sommes de fonctions convexes), donc elle admet un minimum en $\bar{u}$.
\[\Rightarrow L(u,\lambda)\geq L(\bar{u},\lambda)\ \forall u\in\mathbb{R}^n\]
Or, \begin{eqnarray*}
L(\bar{u},\lambda)&=&J(\bar{u})+\underbrace{\lambda^t g(\bar{u})}_{=0} \\
		&=&J(\bar{u})
\end{eqnarray*}
soit $u\in\mathcal{U}_{ad}$. alors :
\begin{eqnarray*}
L(u,\lambda)&=&J(u)+\underbrace{\sum_{i\in I} \underbrace{\lambda_i}_{\geq 0} \underbrace{g_i(u)}_{\leq 0}}_{\leq 0}\\
	&=&J(u)
\end{eqnarray*}
\[J(u)\geq L(u,\lambda)\geq L(\bar{u},\lambda)=J(\bar{u})\]
Donc $\bar{u}$ est un minimum de $J$ sur $\mathcal{U}_{ad}$.
\end{dem}

\subsection{Dualité}
\Def{Point selle}{On dit que $(u,\lambda)\in\mathcal{U}_{ad}\times\mathbb{R}^l_+$ est un point selle de $J$ sur $\mathcal{U}_{ad}\times\mathbb{R}^l_+$ si :
\[\forall \mu\in\mathbb{R}^l_+,\ \forall v\in\mathcal{U}_{ad},\ L(u,\mu)\leq L(u,\lambda)\leq L(v,\lambda)\]}

\Prop{}{Soit $U$ un ouvert contenant $\mathcal{U}_{ad}$ et $(u,\lambda)$ un point selle de $J$ sur $U\times \mathbb{R}_+^l$.Alors $u\in\mathcal{U}_{ad}$ et :
\[\left\{ \begin{array}{c c c}
\nabla_u L(u,\lambda)&=&0\\
\lambda^t g(u)=0
\end{array}\right.\]}

\begin{dem}
Comme $(u,\lambda)$ est un point selle, on a 
\[\forall \mu\in\mathbb{R}^l_+,\ \forall v\in U,\ L(u,\mu)\leq L(u,\lambda)\leq L(v,\lambda)\]
ie :
\[J(u)+\mu^tg(u)\underset{(1)}{\leq} J(u)+\lambda^tg(u) \underset{(2)}{\leq} J(v)+\lambda^tg(v)\]

\[(1)\ :\ \mu^Tg(u)\leq \lambda^Tg(u),\ \forall \mu\in\mathbb{R}^l_+\]
Si on prend $\mu=\frac{1}{2}\lambda$ et $\mu=2\lambda$, on voit bien que $\lambda^tg(u)=0$.\\
Ainsi, $\mu^tg(u)\leq 0$ $\forall\mu\in\mathbb{R}^l_+$.\\
Prenons \[\mu=\begin{pmatrix} 0 \\ \vdots \\ 0 \\ 1 \\ 0 \\ \vdots \\ 0 \end{pmatrix}\]. Ainsi, $g_i(u)\leq 0$ $\forall i\in I$.
\[\Rightarrow u\in\mathcal{U}_{ad}\]

(2) : $u$ est un point de minimum de :
\[u\mapsto J(u)+\lambda^tg(u)=L(u,\lambda)\]
sur U. Donc :
\[\nabla_u L5u,\lambda)=0\]
\end{dem}

\Lem{}{Soit L une fonction de deux variables $u$ et $\lambda$.
	\[\inf_u \sup_\lambda L(u,\lambda)\geq \sup_{\lambda} \inf_u L(u,\lambda)\]}

\begin{dem}
\[\forall u, \sup_{\lambda} L(u,\lambda)\geq \sup_{\lambda} \inf_u L(u,\lambda)\]
Ceci étant vrai pour tout $u$, c'est également vrai pour celui qui minimise le terme de gauche.
\[\Rightarrow \inf_u \sup_{\lambda} L(u,\lambda)\geq \sup_\lambda \inf_u L(u,\lambda)\]
\end{dem}

\Theo{}{On suppose que la contrainte est qualifiée et que $(\mathcal{P}_{\mathcal{U}_{ad}})$ admet une solution.\\
Alors \[\min_{u\in\mathcal{U}_{ad}} J(u)=\sup_{\lambda\in\mathbb{R}^n} \inf_{u\in\mathbb{R}^n} L(u,\lambda)= \inf_{u\in\mathbb{R}^n}\sup_{\lambda\in\mathbb{R}^n} L(u,\lambda)\]
De plus, le problème $\sup_{\lambda\in\mathbb{R}^n} \inf_{u\in\mathbb{R}^n} L(u,\lambda)$ admet une solution $\lambda^*$ et $\inf_{u\in\mathbb{R}^n} \sup_{u\in\mathbb{R}^n} L(u,\lambda)$ admet une solution $u^*$ qui est solution de ($\mathcal{P}_{\mathcal{U}_{ad}}$)}

\Def{}{On note \[d(\lambda)=\inf_{u\in\mathbb{R}^n}\]
Le problème \[\sup_{\lambda\in\mathbb{R}^n} d(\lambda)\]
est le problème dual.}

\begin{dem}
Comme les contraintes sont qualifiées et que le problème admet au moins une solution $u^*$, on a :
\[\left\{ \begin{array}{c c c} 
	\nabla J(u^*) + \sum_{i\in I} \lambda_i^* \nabla g_i(u^*) &=& 0 \\
	\sum_{i\in I} \lambda_i^* g_i(u^*) &=& 0
\end{array}\right. \Leftrightarrow 
\left\{ \begin{array}{c c c}
	\nabla_u L(u^*,\lambda^*)&=&0 \\
	\lambda^{*^t} g(u^*)&=&0
\end{array} \right. \]
Comme $J$, $g_1$, ..., $g_l$ sonr convexes et $\lambda_i^*\geq 0$, on a $u\mapsto L(u,\lambda^*)$ convexe.\\
Donc $u^*$ est un point de maximum de $u\mapsto L(u,\lambda^*)$.
\[\sup_{\lambda\in \mathbb{R}^l_+} \inf_{u\in \mathbb{R}^n} L(u,\lambda) \geq \inf_{u\in \mathbb{R}^n} L(u,\lambda^*)=L(u^*,\lambda^*)\]
De plus,
\begin{eqnarray*}
	L(u^*,\lambda^*)&=&J(u^*)+\underbrace{\sum_{i\in I} \lambda_i^* g_i(u^*)}_{=0} \\
			&=&J(u^*)
\end{eqnarray*}
\[\Rightarrow L(u^*,\lambda^*)=\min_{u\in\mathcal{U}_{ad}} J(u)\leq \sup_{\lambda\in\mathbb{R}^l_+} \inf_{u\in\mathbb{R}^n} L(u,\lambda)\]

\bigskip
Montrons à présent que :
\[\sup_{\lambda\in\mathbb{R}^l_+} L(u,\lambda)=\left\{ \begin{array}{c c c}
	J(u) &\text{ si }& u\in\mathcal{U}_{ad}\\
	+\infty &\text{ sinon}&
\end{array} \right.\]

\begin{itemize}
	\item Si $u\in \mathcal{U}_{ad}$,
	\begin{eqnarray*}
		J(u)=L(u,0)&\leq& \sup_{\lambda\in \mathbb{R}^l_+} L(u,\lambda)\\
				&\leq& J(u)+\sup_{\lambda\in\mathbb{R}^l_+} \sum_{i\in I} \underbrace{\lambda_i}_{\geq 0} \underbrace{g_i(u)}_{\leq 0}\\
				&\leq& J(u)
	\end{eqnarray*}
	\[\Rightarrow \sup_{\lambda\in\mathbb{R}^l_+} L(u,\lambda) = J(u)\]

	\item Si $u\not\in \mathcal{U}_{ad}$, $\exists i$ tel que $g_i(u)>0$. On pose 
	\[\lambda_j^k=\left\{ \begin{array}{c c c} k &\text{ si } i=j \\ 0 &\text{ sinon}& \end{array} \right.\]
	\begin{eqnarray*}
		\sup_{\lambda\in\mathbb{R}^l_+} L(u,\lambda) &\geq& \sup_{k\in \mathbb{N}} L(u,\lambda^k) \\
							&\geq& \sup_{k\in\mathbb{N}} \{J(u)+kg_i(u)\} \\
							&\geq& J(u)+\underbrace{g_i(u)}_{\geq 0} \underbrace{\sup_{k\in\mathbb{N}} \{k\}}_{+\infty}
	\end{eqnarray*}
	\[\Rightarrow \sup_{\lambda\in\mathbb{R}^l_+} L(u,\lambda)=+\infty\]
\end{itemize}

On en déduit que :
\begin{eqnarray*}
	\inf_{u\in\mathbb{R}^n} \sup_{\lambda\in\mathbb{R}^l_+} L(u,\lambda)&=&\inf_{u\in\mathcal{U}_{ad}} J(u) \\
									&=& L(u^*,\lambda^*)\\
									&\leq& \sup_{\lambda\in\mathbb{R}^l_+} \inf_{u\in\mathbb{R}^n} L(u,\lambda)
\end{eqnarray*}
Avec le lemme précédent, on en déduit que :
	\[\sup_{\lambda\in\mathbb{R}^n} \inf_{u\in\mathbb{R}^n} L(u,\lambda)= \inf_{u\in\mathbb{R}^n}\sup_{\lambda\in\mathbb{R}^n} L(u,\lambda)\]
\end{dem}
