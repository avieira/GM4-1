\part{Processus de Markov \`a temps continu}

\section{Générateur infinitésimal}
\subsection{Générateur et équations backward - forward}
$X_t$ : l'état du système à t.\\
On admettra que $\Pi_t$ est dérivable en 0, ie
	\[\forall i,j,\ p_j^i(t) \text{ dérivale en } t=0\]

\Def{}{On notera $A=\Pi_0'=[(p_j^i)'(0)]$. On appelle $A$ le générateur infinitésimal de la chaîne.\\
On notera $a_j^i=(p_j^i)'(0)$, $A=[a_j^i]$ et $a_i=-a_i^i$.}

\underline{Remarque :} \begin{itemize}
	\item $A$ n'est pas une matrice à termes positifs (elle n'est pas stochastique)
	\item L'analogue en temps discret est
		\[\frac{\Pi_1-\Pi_0}{1-0}=\Pi-I\]
\end{itemize}

\Prop{des $a_j^i$}{$\forall i\neq j$, $a_j^i\geq 0$, $a_i^i\leq 0$, et 
\[\sum_j a_j^i=0\]
\[\Rightarrow a_i=\sum_{j\neq i} a_j^i\]}

\begin{dem}
	\[i\neq j,\ \frac{p_j^i(t)-p_j^i(0)}{t}=p_j^i(t)\geq 0 \to a_j^i\geq 0\]
	\[\forall t,\ \sum_j p_j^i(t)=1\Rightarrow \sum_j (p_j^i)'(0)=0=\sum_j a_j^i\]
donc $a_i=-a_i^i=\sum_{j\neq i} a_j^i$.
\end{dem}

\Theo{Backward et Forward}{$\Pi_t$ est dérivable pour tout $t$ et on a :
	\begin{equation} \tag{forward} \Pi_t'=\Pi_tA \end{equation}
	\begin{equation} \tag{backward} \Pi_t'=A\Pi_t \end{equation}
et comme $\Pi_0=I$, on a $\Pi_t=e^{tA}$ (donc la connaissance de $A$ entraine la connaissance des $p_j^i(t)$, $\forall i,j,t$, et inversement).}

\begin{dem}
\[\frac{1}{h}(\Pi_{t+h}-\Pi_t)=\frac{1}{h}(\Pi_t\Pi_h-\Pi_tI)=\Pi_t \left( \frac{\Pi_h-\Pi_0}{h}\right) \xrightarrow[h\to 0]{} \Pi_t A\]
Donc $\Pi_t$ dérivable en $t$ et $\Pi_t'=\Pi_t A$. De même :
\[\frac{1}{h}(\Pi_{t+h}-\Pi_t)=\left(\frac{\Pi_h-\Pi_0}{h}\right)\Pi_t\xrightarrow[h\to0]{}A\Pi_t\]

\bigskip
$e^{tA}$ est inversible.
	\[\left(e^{tA}\right)'=Ae^{tA}=e^{tA}A\]
donc solution de Backwar et Forward, avec $\Pi_0=I$, donc par unicité des solutions, $e^{tA}=\Pi_t$. 
\end{dem}

\underline{Backward :} $(p_j^i)'(t)=\sum_k a_k^ip_j^k(t)$ : c'est le point d'arrivée qui est fixe pour $p_j^k$\\
\underline{Forward :} $(p_j^i)'(t)=\sum_k p_k^i(t)a_j^k$ : c'est le point de départ qui est fixe dans les $p_k^i$.

\subsection{Significations probabilistes des coefficients $a_j^i$ du générateur infinitésimal}
\Def{}{$i\in E$ est absorbant si : \[\forall t,\ p_j^i(t)=1\]}
\Theo{}{i est absorbant si et seulement si : \[\forall j, a_j^i=0 \Leftrightarrow a_i=0\]}

\begin{dem}
Si i est absorbant, alors $p_i^i(t)=1\ \forall t$.\\
Donc $(p_i^i)'(0)=0=-a_i$. Or :
\begin{eqnarray*}
	a_i=0 &\Leftrightarrow& \sum_{i\neq j} a_j^i=0 \text{ et } a_j^i\geq 0\\
		&\Leftrightarrow& \forall j, a^i_j=0
\end{eqnarray*}
Réciproquement, si $a_i=0$, $\forall j$, $a_j^i=0$.
\[(p_i^i)'(t)=\sum_j a_k^ip_i^k = 0\]
Donc $p_i^i(t)=cste=p^i_i(0)=1$ $\forall t$.
\end{dem}

\Def{Instants de transitions}{On dit qu'il y a transition s'il y a changement d'état.
\begin{eqnarray*}
	T_1&=&\inf\{t>0,\ X_t\neq X_0\}\\
	T_2&=&\inf\{t>0,\ X_t\neq X_{T_1}\}\\
	&\vdots&\\
	T_n&=&\inf\{t>0,\ X_t\neq X_{T_{n-1}}\}\\
	&\vdots&
\end{eqnarray*}
Les $T_i$ sont appelés instants de transition. (On pose $T_0=0$)}

\Prop{admise}{\[\forall i, \mathbb{P}(T_n\leq h|X_0=i)=o(h)\]}

\Theo{fondamental}{Si i non abosrbant, alors : \begin{enumerate}
	\item $\mathcal{L}(T_1|X_0=i)=\mathcal{E}(a_i)$
	\item $\forall j\neq i$, $\mathbb{P}(X_{T_1}=j|X_0=i)=\frac{a^i_j}{a_i}=\hat{p}^i_j$
	\item $T_1$ et $X_{T_1}$ sont indépendants sachant que $X_0=i$.
\end{enumerate}}

\begin{dem}
i non absorbant.\\
On va modifir la chaîne de la façon suivante : on rend tous les états $i\neq j$ absorbants. On ne touche pas à la façon d'être en i ni d'en ressortir.\\
On note $X_t^*$ la nouvelle chaîne, et $\Pi_t^*$ et $A^*$ les matrices correspondantes.\\
$(a^*)_j^k=0,\ \forall k\neq i,\ \forall j$ car ils sont absorbants.\\

\bigskip
Prouvons que $(a^*)_j^i=a^i_j\ \forall j$.\\
On a, si on part de i $X_t(\omega)=X^*_t(\omega)$ $\forall t\leq T_2(\omega)$. 
\[\mathbb{P}(X_t=j|X_0=i)=\underbrace{\mathbb{P}(X_t=j,t<T_2|X_0=i)}_{(1)}+\underbrace{\mathbb{P}(X_t=j,t\geq T_2|X_0=i)}_{(2)}\]
$(1)=\mathbb{P}(X^*_t=j|X_0=i)$\\
$0\leq (2)\leq \mathbb{P}(T_2\leq t|X_0=i)=o(t)$\\
Donc $p_j^i(t)=(p^*)_j^i(t)+o(t)$, donc :
	\[a_j^i=(p^i_j)'(0)=({p^*}^i_j)'(0)=(a^*)_j^i,\ \forall j\neq i\]
	\[a_i=a^*_i=\sum_{j\neq i} a_j^i\]
Ainsi :
\[(p^*)_i^i(t)=\mathbb{P}(X_t^*=i|X_0^*=i)=\mathbb{P}(T_1>t|X_0=i)\]
\[\left\{\begin{array}{c c c}
({p^*}^i_i)'(t)&=&\sum_k {p^*}_k^i(t)a_i^k = -a_i{p^*}^i_i(t)\\
{p^*}^i_i(0)=1
\end{array}\right.\]
\[\Rightarrow {p^*}^i_i(t)=\mathbb{P}(T_1>t|X_0=i)=e^{-a_1t}\]
ie $\mathcal{L}(T|X_0=i)=\mathcal{E}(a_i)$.

\bigskip
$j\neq i$. \begin{eqnarray*}
	({p^*}^i_j)'(t)&=&\sum_k {p^*}^i_k(t)a_j^k \\
			&=&{p^*}i^i(t)a_j^i\\
			&=&a_j^ie^{-a_it}
\end{eqnarray*}
${p^*}^i_j(0)=0$\\
Donc : \begin{eqnarray*}
{p^*}^i_j(t)&=&\int_0^t a_j^i e^{-a_it}dt\\
	&=&\frac{a_j^i}{a_i}(1-e^{-a_it})
\end{eqnarray*}

Or,
\begin{eqnarray*}
	{p^*}_i^j(t)&=&\mathbb{P}(X_t^*=j|X_0=i)\\
		&\leq& \mathbb{P}(T_1\leq t, X_T=j | X_0=i)\\
		&\leq& \frac{a_j^i}{a_i}(1-e^{-a_it})
\end{eqnarray*}
Si on prend $t\to +\infty$ :
	\[\mathbb{P}(X_{t_1}=j|X_0=i)=\frac{a_j^i}{a_i}\]
\[\mathbb{P}(T_1\leq t, X_{T_1}=j|X_0=i)=\mathbb{P}(T_1\leq t|X_0=i)\mathbb{P}(X_{T_1}=j|X_0=i)\]
d'où l'indépendance de $X_{T_1}$ et $T_1$ sachant $X_0=i$.
\end{dem}

\subsection{Chaîne discrète de transition associée}
$\hat{X}_n=X_{T_n}$.\\
Elle a pour matrice de transition \[\hat{\Pi}_1=\hat{\Pi}=(\hat{p}_j^i)\] où si i est absorbant :
	\[\hat{p}_i^i=1, \hat{p}_j^i=0,\ \forall j\neq i\]
Si i n'est pas absorbant :
	\[\hat{p}_j^i=\frac{a_j^i}{a_i},\ j\neq i,\ \hat{p}_i^i=0\]
On représnte le diagramme de la chaîne initiale par celui de la chaîne discrète associée.\\
\[j\neq i,\ i\to j \text{ si } \hat{p}_j^i>0 \text{ ie } a_j^i>0\]
Il n'y a pas de boucles (sauf pour les états absorbants, mais ceux-ci sont indiqués par $\Delta$). \\
Ceci donne la classification habitulle des états : Erg, Tr, etc.
\[A=\bordermatrix{&C_1 & C_2 & C_3 & C_4 & C_5\cr
C_1 & A_1 &  0  & |0 & 0 & 0\cr
C_2 &  0  & A_2 & |0 & 0 & 0\cr 
C_3 & \hbox{\raisebox{0.4em}{\vrule depth 0pt height 0.4pt width 1em}} & \hbox{\raisebox{0.4em}{\vrule depth 0pt height 0.4pt width 1em}} & |\hbox{\raisebox{0.4em}{\vrule depth 0pt height 0.4pt width 1em}} &\hbox{\raisebox{0.4em}{\vrule depth 0pt height 0.4pt width 1em}} &\hbox{\raisebox{0.4em}{\vrule depth 0pt height 0.4pt width 1em}} \cr
C_4 &     &  \rho  & |  & \chi &  \cr
C_5 &     &     &  | &  \cr}\]
\[\hat{\Pi}=\bordermatrix{&C_1 & C_2 & C_3 & C_4 & C_5\cr
C_1 & \hat{\Pi}_1 &  0  & |0 & 0 & 0\cr
C_2 &  0  & \hat{\Pi}_2 & |0 & 0 & 0\cr 
C_3 & \hbox{\raisebox{0.4em}{\vrule depth 0pt height 0.4pt width 1em}} & \hbox{\raisebox{0.4em}{\vrule depth 0pt height 0.4pt width 1em}} & |\hbox{\raisebox{0.4em}{\vrule depth 0pt height 0.4pt width 1em}} &\hbox{\raisebox{0.4em}{\vrule depth 0pt height 0.4pt width 1em}} &\hbox{\raisebox{0.4em}{\vrule depth 0pt height 0.4pt width 1em}} \cr
C_4 &     &  \hat{R}  & |  & \hat{Q} &  \cr
C_5 &     &     &  | &  \cr}\]

\[t^nA^n=\begin{pmatrix} \begin{matrix} t^nA^n & & 0 \\ & \ddots & \\ 0 & & t^n A_k^n \end{matrix} & & 0 \\ \\ \rho_n & & t^n\chi^n\end{pmatrix}\]

$e^{tA}=\sum_{n\geq 0} \frac{t^nA^n}{n!}=\Pi_t$
\[\Pi_t=\begin{pmatrix} \begin{matrix} e^{tA_1} & & 0 \\ & \ddots & \\ 0 & & e^{tA_k} \end{matrix} & & 0 \\ \\ R_t & & e^{t\chi}\end{pmatrix}\]

\Theo{}{\[Q=e^{t\chi}\xrightarrow[t\to+\infty]{} 0\]}
\begin{dem}
On prend i transitoire.
\[\mathbb{P}(X_t\in Tr|X_0=i)=\sum_{j\in Tr} \mathbb{P}(X_t=j|X_0=i)\]
C'est décroissant en t.\\
Or on sait que, par exemple, $Q_n=(Q_1)^n\to 0$ (en considérant la chaîne discrète $X_n$).\\
Donc $\mathbb{P}(X_t\in Tr|X_0=i)\to 0$. Donc
	\[Q_t=e^{t\chi}\xrightarrow[t\to+\infty]{}0\]
\end{dem}

\subsection{Temps moyen passé dans les états transitoires}
Soit $s^i_j$ le temps moyen passé par j sachant qu'on est parti de i, i et j transitoire, et $D=[d_j^i]_{i,j\in Tr}$. 
\Theo{}{$\chi$ est inversible et $D=-\chi^{-1}$.}

\begin{dem}
Soit $D_j$ le temps passé en j.
\[D_j=\int_0^{+\infty} \mathds{1}_{\{X_t=j\}}dt\]
\begin{eqnarray*}
d_i^j&=&\mathbb{E}(D_j|X_0=i)\\
	&=&\mathbb{E}(\int_0^{+\infty} \mathds{1}_{\{X_t=j\}}dt|X_0=i)\\
	&=&\int_0^{+\infty} \mathbb{E}(\mathds{1}_{\{X_t=j\}}|X_0=i)dt \text{ (Fubini)}\\
	&=&\int_0^{+\infty} \mathbb{p}(X_t=j|X_0=i)dt\\
	&=&\int_0^{+\infty} p_j^i(t)dt
\end{eqnarray*}
\begin{eqnarray*}
D&=&\int_0^{+\infty} e^{t\chi} dt\\
\chi D&=&\int_0^{+\infty} \chi e^{t\chi} dt\\
	&=&[e^{t\chi}]_0^{+\infty}\\
	&=& -I
\end{eqnarray*}
Donc $\chi$ est inversible et $-\chi^{-1}=D$.
\end{dem}

\Theo{}{\[\forall i,j\in Tr,\ d_j^i=\hat{N}^i_j\times\frac{1}{a_j}\]
où $\hat{N}=(I-\hat{Q})^{-1}$. }

$\hat{N}_j^i$ : nombre de fois où l'on est passé par j sachant que l'on est parti de i.\\
$\frac{1}{a_j}$ : le temps moyen passé en j à chaque séjour (espérance de la loi exponentielle).

\subsection{Probabilité de finir dans une classe finale donnée}
iSoit $b_j^i$ la probabilité que le premier état ergodique atteint soit j sachant que $X_0=i$, et $B=[b_j^i]_{i,j}$ ($i\in Tr$, $j\in Erg$). \\
On sait déjà que (considérant la chaîne associée $\hat{X}_n$) :
	\[B=NR\]
donc
\begin{eqnarray*}
	b_j^i&=&\sum_{k\in Tr} \hat{N}_k^i p_j^k\\
	&=&\sum_{k\in Tr} \hat{N}_k^i \frac{a_j^k}{a_k}\\
	&=&\sum_{k\in Tr} d_k^i a_j^k \\
	&=&(D\rho)^i_j
\end{eqnarray*}

\Theo{}{\[B=\hat{N}\hat{R}=D\rho=-\chi^{-1}\rho\]}

\Theo{}{Soit $C$ une classe finale.\\
La probabilité de finir en $C$ sachant $X_0=i$ est :
	\[b^i_C=\sum_{j\in C} b^i_j\]}

\subsection{Classification de la chaîne}
\[i\to j \Rightarrow \hat{p}_j^i>0\Leftrightarrow \frac{a_j^i}{a_i}>0 \Leftrightarrow a_j^i>0\]
i peut conduire à j s'il existe un chemin orienté allant de i ) j :
	\[\exists n;\ \mathbb{P}(X_{T_n}=j|X_0=i)>0\]

\Theo{}{Soient $i,j\in E$. On a équivalence entre les trois points suivants :
\begin{enumerate}
	\item $i\rightsquigarrow j$
	\item $\exists t>0;\ p_j^i(t)>0$
	\item $\forall t>0;\ p_j^i(t)>0$
\end{enumerate}}

\begin{dem}
Il est évident que $3\Rightarrow 2\Rightarrow 1$. Prouvons que $1\Rightarrow 3$.\\
Supposons d'abord que $i\to j$ ($a_j^i>0$). Montrons que $\forall t>0$, $p_j^i(t)>0$.
\begin{eqnarray*}
	(p_i^i)'(t)&=&\sum_k p_k^i(t)a_i^k\\
		&=& p_i^i(t)a_i^i + \underbrace{\sum_{k\neq i} p_k^i(t)a_i^k}_{R_i(t)\geq 0}
\end{eqnarray*}
\[\left\{\begin{array}{c c c}
(p_i^i)'(t)&=&-a_ip_i^i(t)+R_i(t)\\
p_i^i(0)=1
\end{array}\right.\]
\[\Rightarrow p^i_i(t)=\underbrace{e^{-a_i t}}_{>0} + \underbrace{\int_0^t e^{-a_i(t-s}R_i(s) ds}_{\geq 0}\]
Donc $\forall i,\ \forall t\geq 0, p_i^i(t)>0$. 

\bigskip
Maintenant, $\forall j\neq i$,
\begin{eqnarray*}
	(p_j^i)'(t)&=&\sum_k p_k^i(t)a_j^k\\
		&=& p_j^i(t)a_j^j +\underbrace{a_i^j p_i^i(t)}_{>0} + \underbrace{\sum_{k\neq i,j} p_k^i(t)a_j^k}_{R_i(t)\geq 0}
\end{eqnarray*}
\[\left\{\begin{array}{c c c}
(p_i^j)'(t)&=&-a_jp_j^i(t)+R_i(t)\\
p_i^i(0)=1
\end{array}\right.\]
\[\Rightarrow p^j_i(t)=\int_0^t e^{-a_i(t-s}\underbrace{R_i(s)}_{\geq 0} ds\]
Donc $\forall i,\ \forall t\geq 0, p_j^i(t)>0$. 

\bigskip
Si $i\rightsquigarrow j$ en n transitions :
	\[i_0\to ... \to i_{n-1}\to i_n=j\]
un tel chemin permet d'aller de i à j.
\begin{eqnarray*}
p_j^i(t)&=&\mathbb{P}(X_t=j|X_0=i)\\
	&\geq&\mathbb{P}(X_t=i_n,X_{\frac{(n-1)t}{n}}=i_{n-1},...,X_{\frac{t}{n}}=i_1|X=i_0=i)\\
	&\geq&\mathbb{P}(X_t=i_n|X_{\frac{(n-1)t}{n}}=i_{n-1})\times...\times\mathbb{P}(X_{\frac{t}{n}}=i_1|X=i_0=i) \text{ d'après la propriété de Markov homogène}\\
	&\geq&\mathbb{P}(X_\frac{t}{n}=i_n|X_0=i_{n-1})\times...\times\mathbb{P}(X_{\frac{t}{n}}=i_1|X=i_0=i)\\
	&>& 0
\end{eqnarray*}
D'où $p_j^i(t)>0$
\end{dem}
Donc $i\rightsquigarrow j\Leftrightarrow \forall t>0,\ p_j^i(t)>0$ (car encore vrai si i absorbant ou si i=j).

\underline{Remarque :} Il n'y a donc pas de phénomène cycliques en temps continu.

\section{Chaîne régulière}
\Def{}{En temps continu, une chaîne est dite régulière si elle n'a qu'une seule classe (donc finale), ie $\forall i,j\in E, i\rightsquigarrow j$}

\Theo{}{La chaîne est régulière si et seulement si : \[\forall t>0,\ \forall i,j,\ p_j^i(t)>0\]}

\subsection{Loi de probabilité invariante}
$\mu$ est invariante $\Leftrightarrow$ $(\mathcal{L}(X_0)=\mu\Rightarrow \forall t>0,\ \mathcal{L}(X_t)=\mu)$, ie \[\forall t, \mu\Pi_t=\mu\]

\Theo{}{Soit $\mu$ une loi de probabilité sur E. \[\mu \text{ est invariante } \Leftrightarrow \mu A=0\]}
\begin{dem}
\[\forall t,\ \mu \Pi_t=\mu\Rightarrow (\mu\Pi_t)'=\mu\Pi_t'=0\]
En particulier, pour $t=0$, $\mu A=0$

\bigskip
Réciproquement, si $\mu A=0$, alors $\mu \Pi'_t=(\mu\Pi_t)'=\mu A\Pi_t=0$, ie
	\[\forall t, \mu\Pi_t=\mu \Leftrightarrow \mu A=0\]
\end{dem}

Donc $\mu$ est une loi de probabilité invariante 
	\[\Leftrightarrow \left\{ \begin{array}{r c c} 
		\mu A&=&0\\
		\forall i, \mu_i&\geq& 0\\
		\sum_i \mu_i &=&1
	\end{array} \right.\]

\Theo{fondamental pour les chaînes régulières (E fini)}{Si la chaîne est régulière, alors elle possède une et une seule loi de probabilité invariante.\\
Notons la $\mu$. Celle-ci vérifie : \begin{enumerate}
	\item $\forall i, \mu_i>0$
	\item $\forall \mathcal{L}(X_0)$, $\mathcal{L}(X_t)\to \mu$ exponentiellement vite
	\item $\mu$ est l'unique solution de $\left\{ \begin{array}{c c c} \mu A&=&0 \\ \sum_i \mu_i &=& 1 \end{array}\right.$
\end{enumerate}}

\Rap{Lemme en temps discret}{$\Pi$ matrice stochastique, $p^*=\inf_{i,j} p^i_j$. \\
X : vecteur colonne : $(X^i)_i$.\\
\[Y=\Pi X\]
$M_1 = \sup_i Y^i$, $M_0=\sup_i X^i$, \\
$m_1=\inf Y^i$, $m_0=\inf X^i$.
On a alors : 
	\[(M_1-m_1)\leq (1-2p^*)(M_0-m_0)\]}

\begin{dem}
Considérons la colonne $j$ de $\Pi_t$ :
	\[C_{t,j}=[p^i_j(t)]_i\]
$j$ sera fixé, on omettra donc l'indice $j$ pour $C_{t,j}$.
\[C_t=\Pi_tC_0,\ C_0=[\delta_j^i]_i\]
\[C_{t+s}=\Pi_t\pi_sC_0=\Pi_tC_s\]
$M_t=\sup_i p_j^i(t)$, $m_t=\inf_i p_j^i(t)$, $p^*(t)=\inf_{i,j} p^i_j(t)$.

	\[\forall t>0, p^*(t)>0 \text{ (car la chaîne est régulière)}\]
Choisissons un $t_0>0$.
	\[t=kt_0+r,\ 0\leq r\leq t_0\]
	\[\Pi_{kt_0+r}=\Pi_{kt_0}\Pi_r \Rightarrow C_{kt_0+r}=\Pi_{t_0}C_{(k-1)t_0+r}\]

\begin{eqnarray*}
M_{kt_0+r}-m_{kt_0+r}&\leq& (1-2p^*(t_0))(M_{(k-1)t_0+r}-m_{(k-1)t_0+r})\\
			&\leq& (1-2p^*(t_0))^k\underbrace{(M_r-m_r)}_{\leq 1}\\
			&\leq& \left[(1-2p^*(t_0))^\frac{1}{t_0}\right]^{kt_0}\\
			&\leq& \left[(1-2p^*(t_0))^\frac{1}{t_0}\right]^{kt_0+r}\times \frac{1}{(1-2p^*(t_0))^\frac{r}{t_0}}
\end{eqnarray*}

Soit : \[a=\sup_{0\leq r<t_0} \frac{1}{(1-2p^*(t_0))^\frac{r}{t_0}}\]
	\[e^{-b}=(1-2p^*(t_0))^\frac{1}{t_0}\]

On a :
	\[M_t-m_t=ae^{-bt}\]
Cela valant pour toute colonne j (implicite depuis le début).

\bigskip
On a $m_t$ croissante et $M_t$ décroissante, avec $0<m_t\leq M_t$. Donc $m_t$ et $M_t$ convergent vers une même limite. Notons $\mu_j$ cette limite.
	\[\forall t,\ °<m_t\leq \mu_j\leq M_t\]
donc $\forall j,\ \mu_j>0$.\\
De plus, $0<m_t\leq p^i_j(t)\leq M_t$
	\[\Rightarrow \forall i,j, |\mu_j-p_j^i(t)|\leq M_{t,j}-m_{t,j}\leq ae^{-bt}\]
donc $p^i_j(t)\to \mu_j$ exponentiellement vite.

\bigskip
Soit $\mu_0=(\mathbb{P}(X_0=j))_j$ la loi initiale, alors $\mu_t=(\mathbb{P}(X_t=j))_j$.
	\[\mathbb{P}(X_t=j)=\sum_i \mu_0(i)p^i_j(t)\]
\begin{eqnarray*}
	|\mathbb{P}(X_t=j)-\mu_j|&\leq& \sum_i \mu_0(i)|p^i_j(t)-\mu_j| \\
				&\leq& \underbrace{\sum_i \mu_0(i)}_{=1} ae^{-bt}
\end{eqnarray*}
$\forall \mathcal{L}(X_0)$, $\mathcal{L}(X_t)\to \mu$ exponenitellement vite.

\bigskip
Montrons pour finir que $\mu$ est une loi de probabilité invariante.
	\[\mu=(\mu_j)_j \text{ vecteur ligne}\]
$\forall j,\ \mu_j>0$.
	\[1=\sum_j p^i_j(t) \to \sum_j \mu_j \Rightarrow \sum_j \mu_j=1\]
	\[\mu\Pi_t \xrightarrow[t\to +\infty]{} \mu\]
	\[\mu \Pi_{t+s} \to \mu\]
	\[\mu\Pi_t\Pi_s \xrightarrow[t\to+\infty]{} \mu\Pi_s\]
Donc $\forall s$, $\mu\Pi_s=\mu$.

\bigskip
Soit $\nu$ tel que $\begin{array}{c c c} \nu A&=&0 \\ \sum_i \nu_i&=&1 \end{array}$.
	\[(\nu \Pi_t)'=\nu A\Pi_t=0\]
	\[\nu\Pi_t=\nu\Pi_0=\nu\ \forall t\]
	\[(\nu\Pi_t)_j=\sum_i \nu_ip^i_j(t)\to \sum_i \nu_i\mu_j = \mu_j\]
donc :
	\[\nu=\nu\Pi_t\to \mu\]
donc $\nu=\mu$.\\
Soit $\nu$ une loi de probabilité invariante. Alors $\nu A=0$ et $\sum_i \nu_i=1$ $\Rightarrow$ $\nu=\mu$.\\
$\mu$ est donc l'unique loi de probabilité invariante sur E.
\end{dem}

\Theo{ergodique}{Si $E$ n'a qu'une classe finale qu'on notera X, et $\mu_C$ la loi de probabilité invariante associée à C, alors $\forall f:E\to \mathbb{C}$ :
	\[\frac{1}{T}\int_0^T f(X_s)fs \xrightarrow[\forall \mathcal{L}(X_0)]{p.s.} \int_C fd\mu_C=\sum_{j\in C} f(j)\mu_C(j)\]}
