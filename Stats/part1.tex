\part{Analyse descriptive des s\'eries chronologiques}
\section{Notations}
\Def{Série chronologique}{Suite finie de données quantitatives indexée par le temps.\\
Si on considère une série chronologique de longueur $n$ :
\begin{itemize}
	\item $t_1$,...,$t_n$ désigne les $n$ instants successifs d'observation
	\item $y_i$ sera la valeur mesure à l'instant $t_i$ (en considérant les dates d'observations équidistantes).
\end{itemize}}

\section{Modèles de décomposition déterministes}
Deux modèles sont étudiés :
\begin{enumerate}
	\item Le modèle additif
	\item Le modèle multiplicatif
\end{enumerate}
combinant chacun :
\begin{enumerate}
	\item Une tendance $f_i$
	\item Une composante saisonnière $s_i$
	\item Une composante résiduelle $e_i$
\end{enumerate}	

\Def{Modèle additif}{Le modèle additif prédit une étiquette sous la forme suivante :
\[y_i=f_i+s_i+e_i,\ i=1..n\]
avec :
\[\sum_{j=1}^p s_j = 0 \text{ et } \sum_{i=1}^n e_i=0\]
Où $p$ désigne une période.\\
On utilise ce modèle quand, en reliant minima et maxima, on obtient deux droites parallèles.}

\Def{Modèle additif}{Le modèle multiplicatif prédit une étiquette sous la forme suivante :
\[y_i=f_i(1+s_i)(1+e_i),\ i=1..n\]
avec :
\[\sum_{j=1}^p s_j = 0 \text{ et } \sum_{i=1}^n e_i=0\]
Où $p$ désigne une période.\\
On utilise ce modèle quand, en reliant minima et maxima, on obtient une sorte de cône.}

\section{Ajustement de la tendance}
\subsection{Ajustement linéaire}
\Formu{Méthode des moindres carrés}{Elle vient de la recherche des paramètres $a,b\in\mathbb{R}$ minimisant la fonctionnelle suivante :
\[\sum_{i=1}^n\left( y_i-(at_i+b)\right)^2\]
ce qui nous donne :
\begin{eqnarray*}
\hat{a}&=&\frac{\sum_{i=1}^n (y_i-\bar{y})(t_i-\bar{t})}{\sum_{i=1}^n (t_i-\bar{t})^2}\\
\hat{b}&=&\bar{y}-\hat{a}\bar{t}
\end{eqnarray*}}

\Formu{Méthode des deux points}{Cette méthode consiste à choisir arbitrairement deux points par lesquels on fait passer une droite.\\
La réalisation de cette méthode se fait en général en prenant deux sous-suites, et en prenant les points médians de chaque sous-série.\\
Cette méthode s'avère efficace en présence de points aberrants, chose que la méthode des moindres carrés ne prend pas en compte.}

\Prop{Appréciation des régression linéaire}{Un moyen de qualifier la qualité de la regression linéaire est d'utiliser le coefficiet de corrélation linéaire, noté $r$, et défini par :
\[r=\frac{\text{cov}(y,t)}{\sigma_y\sigma_t}\]
En effet, en réécrivant l'expression, on peut montrer que :
\[r^2=\frac{\text{Variance expliquée}}{\text{Variance totale}}\]}

\subsection{Ajustement polynomial}
\Formu{Polynôme des moindres carrés}{On minimise la même fonction que précédemment, mais en cherchant cette fois non plus $a$ et $b$ d'une régression linéaire mais $a_i,\ i=0,..,d$ d'un polynôme de degré $d$. En notant :
\[Y=\begin{pmatrix} y_1 \\ \vdots \\ y_n \end{pmatrix} \text{ et } T=\begin{pmatrix} 1 & t_1 & \cdots & t_1^d \\ \vdots & \vdots & \ddots & \vdots \\ 1 & t_n & \cdots & t_n^d\end{pmatrix}\]
On obtient :
\[\theta^{MC}=\begin{pmatrix} a_0 \\ \vdots \\ a_d \end{pmatrix}=({}^tTT)^{-1}\times{}^tTY\]
}

\subsection{Ajustement non linéaire}
On a deux cas :
\begin{enumerate}
	\item Soit on se ramène à un ajustement linéaire via un changement de variable
	\item Soit on cherche à déterminer les coefficients restants via une méthode à $d$ points ($d$ étant le nombre de paramètres à estimer), ou en minimisant le carré des erreurs (ce qui ne donne pas toujours une formule explicite). 
\end{enumerate}

\section{Lissage par moyenne mobile}
\Def{Moyenne mobile simple}{On note $MM(k)$ la série des moyennes mobiles d'ordre $k$ de la série $(y_j)_{j=1...n}$, et on a :
\begin{itemize}
\item lorsque $k$ est pair et vaut $2m$ : \[MM(k)_j=\frac{y_{j-m+1}+...+y_j+y_{j+1}+...+y_{j+m}}{2m}\]
\item lorsque $k$ est impair et vaut $2m+1$ : \[MM(k)_j=\frac{y_{j-m}+...+y_j+y_{j+1}+...+y_{j+m}}{2m+1}\]
\end{itemize}
pour $j=m+1,...,n-m$.}

\Def{Moyenne mobile centrée}{La série notée $MMC(k)$ uniquement pour $k$ pair et définie par :
\[MMC(k)_j=\frac{MM(k)_{j-1}+MM(k)_j}{2}\]}

\Prop{}{\begin{itemize}
\item La série $MM(p)$ ou $MMC(p)$ ne possède plus de composante saisonnière de période $p$.
\item Une moyenne mobile atténue l'aplitude des fluctuations irrégulières d'une chronique.
\end{itemize}}

\section{Décomposition d'une série chronologique}
\Formu{Étapes de la décomposition}{\begin{enumerate}
\item La désaisonnalisation
	\begin{enumerate}
	\item Lissage par moyennes mobiles : on construit la série des moyennes mobiles d'ordre p, la saisonnalité (centrées si $p$ pair).
	\item Constrution de la série des différences / quotients : observation - série des moyennes mobiles ou obs / MM
	\item Calcul des coefficients saisonniers non centrés : moyennes des différences pour chaque saison
	\item Centrage des coefficients saisonniers : moyennes des $p$ coefficients non centrés, puis on centre les coefficients saisonniers.
	\item Constrution de la série corrigée des variations saisonnières : observation - composante saisonnière (selon, bien sûr, la saison) ou division.
	\end{enumerate}
\item La série lissée des prévisions
	\begin{enumerate}
	\item Ajustement d'une tendance : regression linéaire (ou autre) sur la CVS
	\item Construction de la série lissée des prévisions : résultat de la régression + coefficient saisonnier. = $\hat{y}_i$ (ou = $\hat{f}_i(1+\hat{s}_i)$)
	\end{enumerate}
\end{enumerate}}
