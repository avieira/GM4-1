\part{\'El\'ements finis}
\textit{Problème modèle :} On va considérer une EDP elliptique (basée sur le laplacien $\Delta$, somme des dérivées secondes)
\[(P)\left\{\begin{array}{c c c}
-u''(x) &=& f(x) \\
u(0)&=&u(1)=0
\end{array}\right.\ \forall x\in\Omega=]0,1[  \]

\section{Formulation variationnelle}
Cette formulation permet de "baisser" l'ordre de dérivation (via la formule de Stroke ou une IPP).
\subsection{Choix de l'espace}
On va définir l'espace V :
\[V=\{u\in L^2(\Omega),\ u'\in L^2(\Omega),\ \underbrace{u(0)=u(1)=0}_{\text{Conditions de Dirichlet}}\}\]

\underline{Remarque :} On a intégré les conditions de Dirichlet homogènes dans la définition de V.

\bigskip
On notera $V=H_0^1(\Omega)$ un espace de Sobolev, qui est un espace de Hilbert.\\
On définit :
\[\langle u,v\rangle=\int_{\Omega}uv dX + \int_{\Omega} u'v'dX\]
\[\|v\|^2=\langle v,v\rangle = \|v\|^2_{L^2(\Omega)} + \|v'\|^2_{L^2(\Omega)}\]

\subsection{Recherche de solution}
On cherche la solution $u(x)$ dans V. $\forall v\in V$ (appelée fonction test) :
\begin{eqnarray*}
	-u''=f \\
	-vu''=vf \\
	-\int_{\Omega} u''v dx = \int_{\Omega} fv dx
\end{eqnarray*}

On a de plus :
\[-\underbrace{[u'v]_0^1}_{=0 \text{ car } v\in V} + \int_0^1 u'v' dx = \int_{\Omega} fv dx\]

On se ramène donc au problème suivant ; trouver $u\in V$; $\forall v\in V$ :
\[(P.V.) \left\{ \begin{array}{c c}
a(u,v)=L(v)\\
\text{avec } & a(u,v)=\int_0^1 u'v' x \\
	     & 	L(v)=\int_0^1 fv dx
\end{array}\right.\]

La solution de PV est appelée solution faible \\
La solution de P est appelée solution forte.

\subsection{Existence-unicité d'une solution de (PV)}

\Theo{de Lax-Milgram}{$a(\bullet,\bullet)$ est :
\begin{itemize}
	\item une forme bilinéaire (symétrique ?)
	\item V-elliptique : $a(u,u)\geq \alpha \|u\|_v^2,\ \alpha\geq 0$
	\item continue : $|a(u,v)| \leq C\|u\|_V\|v\|_V,\ C\geq 0$
\end{itemize}
$L(\bullet)$ est linéaire continue.\\
Sous ces conditions, (PV) admet une solution unique $u\in V$.}

\section{Approximation numérique du problème variationnel}
C'est dans cette partie que l'on va utiliser la méthode des élements finis en exprimant la solution discrétisée $u_h$ dans une base d'un espace $V_h$ de dimension finie.

\subsection{Choix de $V_h$}
Choix le plus simple :
\[V_h=\{v_n\in V; v_h\in\mathcal{C}^0(\bar\Omega), \forall i=0,...,n-1;\ {v_h}_{[\tau_i,\tau_{i+1}]}\in\mathbb{R}_1[X]\}\]

$V_h$ est un espace vectoriel de dimension $n-1$. 

\bigskip
\underline{Remarque :} Cela correspond à des $\beta$-splines.
\[\tau=(\tau_i)_{i=0..n},\ \dim \mathbb{P}_{k,\tau,r}=(k+1)n-\sum_{i=1}^{n-1} r_i\]
$\mathbb{P}_{k,\tau,r}$ est l'espace des fonctions polynomiales par morceaux de degré inférieur ou égal à k avec un raccord $\mathcal{C}^{r_i-1}$ en $\tau_i$.

\bigskip
En particulier, $\dim V_h=n-1$.

\subsection{Fonctions de base}
Ce sont les $(\phi_i)_{i=1,...,n-1}$, ils vérifient une condition lagrangienne :
\[\left\{\begin{array}{c c c c}
\phi_i(z_i)&=&1 & \\
\phi_i(z_j)&=&0\ & \forall j\neq i
\end{array}\right.\]

\subsection{Problème discrétisé}
Le problème discrétisé $(PV_h)$ est maintenant la recherche de $u_h(x)=\sum_{j=1}^{n-1} \xi_i \phi_i(x)\in V_h$ tel que $\forall v_h\in V_h, a(u_n,v_n)=L(v_n)$.

\subsection{Détermination des inconnues $(\xi_j)_{j=1,...,n-1}$}
On a :
\[u_h(\tau_i)=\sum_{j=1}^{n-1} \xi_j \phi_j(\tau_i)=\xi_i\]
$(PV_h)$ est équivalent à : \\
Trouver $u_h(x)=\sum_{j=1}^{n-1}\xi_j \phi_j(x)\in V_h$ tel que $\forall i=1,...,n-1$, $a(u_h,\phi_i)=L(\phi_i)$. 

On a fonc:
\[\sum_{j=1}^{n-1} \xi_j \underbrace{a(\phi_j,\phi_i)}_{\text{Matrices de rigidité}}=L(\phi_i),\ \forall i=1,...,n-1\]

On se ramène donc à un système linéaire :
\[R\xi=F\]

\Prop{}{R est définie positive.}

\begin{dem}
On l'obtient grâce à la V-ellipticité de $a(\bullet,\bullet)$. \\
Soit $v\in\mathbb{R}^{n+1}$. On cherche à voir si $v^TRv>0$.

\[(v^TR)_j=\sum_{i=1}^{n-1} v_iR_{ij}=\sum_{i=1}^{n-1} v_ia(\phi_j,\phi_i)=a(\phi_j,\sum_{i=1}^{n-1} v_i\phi_i)\]
\[(v^TR)v=\sum_{j=1}^n v_j(v^TR)_j = a\left( \sum_j v_j\phi_j, \sum_i v_i\phi_i\right)\]

Posons $w=\sum_i v_i\phi_i$, $a(w,w)\geq 0$ car $a(\bullet,\bullet)$ V elliptique. 
\end{dem}

\part{Petits rappels d'analyse fonctionnelle}
\section{Rappels sur les distributions}
\underline{Notation :} $\alpha=(\alpha_1,...,\alpha_n)\in\mathbb{N}^*$, on définit :
\[\partial^\alpha=\frac{\partial^{|\alpha|}}{\partial^{\alpha_1}x_1...\partial^{\alpha_n}x_n}\]
avec $|\alpha|=\sum_{i=1}^n \alpha_i$

\Prop{}{\begin{itemize}
\item $\mathcal{D}(\Omega)$ est dense dans $L^2(\Omega)$ (toute fonction de $L^2(\Omega)$ est limite d'une suite de fonctions incluse dans $\mathcal{D}(\Omega))$.
\item L'application identité de $L^2(\Omega)$ dans $\mathcal{D}'(\Omega)$ est appelée injection canonique. Elle est continue.
\item $f_n\xrightarrow{L^2(\Omega)} f \Rightarrow T_{f_n} \xrightarrow{\mathcal{D}'(\Omega)} T_f$
\end{itemize}}

\section{Espaces de Sobolev}
Ces espaces nous permettent de résoudre les problèmes variationnels. Les espaces de Sobolev se construisent à partir des espaces $L^p$ (on va d'abord s'intéresser aux espaces $H^m(\Omega)$ construits sur $L^2(\Omega)$).

\subsection{Liens entre $\mathcal{D}(\omega)$, $L^2(\Omega)$ et $H^1(\Omega)$}
On rappelle la notion de dérivation faible :
	\[u\in L^2(\Omega),\ \frac{\partial u}{\partial x_i} \to \omega_i \in \mathcal{D}'(\Omega)\]
\Def{Espaces des Sobolev}{Les dérivées qui vont intervenir dans les espaces de Sobolev sont prises au sens des distributions. 
	\[H^1(\Omega)=\left\{v\in L^2(\Omega),\ \frac{\partial v}{\partial x_i}\in L^2(\Omega),\ \forall i=1,...,n\right\}\]
où $\Omega\subset\mathbb{R}^n$. \\
On définit un produit scalaire :
\begin{eqnarray*}
	((u,v))_{1,\Omega}&=&\int_{\Omega} \left(uv+\sum_{i=1}^n \frac{\partial u}{\partial x_i} \frac{\partial v}{\partial x_i}\right) dx\\
			&=&\int_{\Omega} \left( uv + \left(\nabla u\right)^t \nabla v\right) dx
\end{eqnarray*}

et on note $\|\bullet\|_{1,\Omega}$ sa norme associée.}

\Prop{}{\begin{itemize}
\item $H^1(\Omega)$ est un espace de Hilbert
\item $H^1(\Omega)$ est séparable (il existe une partie dénombrable et dense dans $H^1(\Omega)$).
\end{itemize}}

\begin{dem}
On va montrer que $H^1(\Omega)$ est complet.\\
Soit $(v_p)_p$ une suite de Cauchy dans $H^1(\Omega)$. On a : $\forall p,\ v_p\in H^1(\Omega)$ et :
\[\forall \varepsilon>0, \exists N\in\mathbb{N}; \forall n,p\geq N, \|v_n-v_p\|<\varepsilon\]
Par définition de $H^1(\Omega)$, $(v_p)_p$ est une suite de Cauchy dans $L^2(\Omega)$ et $\forall i=1,...,n$, $\left(\frac{\partial v_p}{\partial x_i} \right)_p$ est également une suite de Cauchy dans $L^2(\Omega)$.\\
\[\exists v\in L^2(\Omega); v_p\xrightarrow{L^2} v\]
\[\exists w_i\in L^2(\Omega); \frac{\partial v_p}{\partial x_i}\xrightarrow{L^2} w_i\]
car $L^2(\Omega)$ est complet.

\bigskip
On rappelle que la convergence dans $L^2(\Omega)$ implique la convergence dans $\mathcal{D}(\Omega)$ (car les fonctions de classe $\mathcal{C}^\infty$ à support compact sont $L^2$ et le produit scalaire de $L^2$ coincide avec le crochet de dualité au sens des distributions).

\bigskip
La convergence se fait donc au sens des distributions. Or, les opérations de dérivation sont continues dans $\mathcal{D}'(\Omega)$. Par conséquent :
\[\frac{\partial v_p}{\partial x_i} \to \frac{\partial v}{\partial x_i}\]
et de plus, il y a unicité de la limite dans $\mathcal{D}'(\Omega)$ et donc $\omega_i=\frac{\partial v}{\partial x_i}$.
\end{dem}

\Prop{Rellich}{Soit $\Omega$ un ouvert borné de $\mathbb{R}^n$ à frontière "suffisament régulière", alors de toute suite bornée dans $H^1(\Omega)$, on peut extraire une sous-suite qui converge dans $L^2(\Omega)$.\\
(L'injection canonique de $H^1$ dans $L^2$ est compacte)}

\Def{}{On désigne par $H_0^1(\Omega)$ l'adhérence de $\mathcal{D}(\Omega)$ dans $(H^1(\Omega), \|\bullet\|_{1,\Omega})$
\begin{eqnarray*}
	H_0^1(\Omega)&=&\{f\in H^1(\Omega) ; \exists \phi_n\in\mathcal{D}(\Omega); \phi_n\to f\}\\
		&=&\left\{u\in L^2(\Omega); \frac{\partial u}{\partial x_i}\in L^2(\Omega); u\left(\Gamma\right)=\{0\}\right\}
\end{eqnarray*}}

\Prop{}{\begin{itemize}
\item $\mathcal{D}(\Omega)$ est dense dans $H_0^1(\Omega)$
\item $(H^1_0(\Omega), \|\bullet\|_{1,\Omega})$ est un Hilbert.
\end{itemize}}

\Formu{de Poincaré}{Si $\Omega$ est borné au moins dans une direction, alors $\exists C(\Omega)>0; \forall v\in H_0^1(\Omega)$;
\[\|v\|_{L^2(\Omega)}=\|v\|_{0,\Omega}\leq C(\Omega)\sum_{i=1}^n \left\| \frac{\partial v}{\partial x_i}\right\|_{L^2(\Omega)}\]}

\begin{dem}
A reprendre
\end{dem}

Il existe d'autres formules de ce type, comme la formule de Poincaré-Wirtinger.

\bigskip
\underline{Remarque :} On munit $H_0^1(\Omega)$ de la norme induite par $H^1(\Omega)$.\\
$H_0^1(\Omega)$ est fermé dans $H^1(\Omega)$ $\Rightarrow$ $H_0^1(\Omega)$ est de Hilbert.

\Prop{}{Si $\Omega$ est borné, alors sur $H_0^1(\Omega)$, la semi-norme \[\left( \int_\Omega (\nabla u)^2 du\right)^{\frac{1}{2}}\] définit une norme équivalente à $\|\bullet\|_{H^1(\Omega)}$}

\begin{dem}
D'après l'inégalité de Poincaré : $\Omega$ borné, $v\in H_0^1(\Omega)$ :
\begin{eqnarray*}
	\int_{\Omega} (\nabla v)^2 dx &\geq& C(\Omega) \int_\Omega v^2 dx \\
	2\int_\Omega (\nabla v)^2 dx &\geq& C(\Omega) \int_\Omega v^2 dx + \int_\Omega (\nabla v)^2 dx \\
	\|v\|_{1,\Omega}^2 = \int_\Omega (\nabla v)^2 dx &\geq& \frac{1}{2} \min (1,C(\Omega)) \|v\|^2_{1,\Omega}
\end{eqnarray*}
\end{dem}

\section{Théorèmes de trace}
On suppose $\Omega$ "régulier", alors $\mathcal{D}(\overline{\Omega})$ est denste dans $H^1(\Omega)$ et l'application 
\[\begin{array}{c c c}\gamma_0 : \mathcal{D}(\overline{\Omega}) &\to& L^2(\Gamma) \\
					v			&\mapsto& \gamma_0 v = v_{|\Gamma} 
\end{array}\]
se prolonge par continuité en une application linéaire continue de $H^1(\Omega)$ dans $L^2(\Gamma)$.

\bigskip
$L^2(\Gamma)$ : classe de fonctions de carré sommable avec la mesure $d\sigma$ (qui est la mesure superficielle sur $\partial \Omega = \Gamma$, associé à la mesure classique de Lebesgue).

\bigskip
\underline{Remarque :} $\gamma_0$ n'est pas surjective (preuve dans la littérature : Allaire, Brégis...)

\section{Généralisation de Sobolev}
$\Omega$ : ouvert non vide e $\mathbb{R}^n$.

\Def{}{On note $W^{m,p}(\Omega)$ $(1\leq p\leq \infty)$ l'espace des fonctions $v\in L^p(\Omega)$ telles que pour tout $\alpha\in \mathbb{N}^n$ tel que $|\alpha|\leq m$, les dérivées partielles $\partial ^\alpha v$ de longueur $|\alpha|$ soient $\mathcal{C}^p(\Omega)$.}

\[\|v\|^2_{m,p,\Omega} = \sum_{|\alpha|\leq m} \int_\Omega (\partial^\alpha v)^2 dx \text{ si } 1\leq p < \infty\]
On a aussi une semi norme :
\[|v|^2_{m,p,\Omega} = \sum_{|\alpha|=m} \int_\Omega (\partial^\alpha v)^2 dx \]

\bigskip
\underline{Remarque :} Lorsque $p=2$, on retombe sur $H^m(\Omega)$

\Def{Fonction $\mu$-holderiennes}{On note $C^{m,\mu}(\overline{\Omega})$ l'espace des fonctions $v$ de $\mathcal{C}^m(\overline{\Omega})$ qui sont $\mu$-holderiennes sur $\overline{\Omega}$, ainsi que toutes leurs dérivées partielles d'ordre $|\alpha|\leq m$, ie :
\[\exists C>0;\ \forall x,y\in \overline{\Omega}, \forall |\alpha|\leq m, |\partial^\alpha v(x) - \partial^\alpha v(y)|\leq C \langle x-y\rangle^\mu_{\mathbb{R}^n}\]
avec $\langle \bullet \rangle$ la norme euclidienne sur $\mathbb{R}^n$. }

Nous allons maintenant donner quelques résultats de compacité dans les espaces de Sobolev :
\[H^m(\Omega) \hookrightarrow \mathcal{C}^s(\Omega) \text{ si } m> s+\frac{n}{2}\]
où $\Omega\subset \mathbb{R}^n$ à frontière lipschitzienne.\\
$\hookrightarrow$ : injection canonique.

\section{Quelques résultats essentiels en analyse hilbertienne}
Soit $H$ un espace de Hilbert muni du produit scalaire $(\bullet, \bullet)_H$. On note $H'$ le dual de $H$.
	\[\|l\|_{H'} = \sup_{v\in H} \frac{|l(v)|}{\|v\|_H}\]

\Theo{de projection}{Soit $K$ un espace convexe, fermé et non vide de $H$. Alors pour tout $f\in H$, il existe un unique élément de $K$, noté $P_K f$ tel que :
	\[\|f-p_kf\|_H = \min_{v\in K} \|f-v\|_H\]}

\underline{Remarque :} $P_k$ est une contraction

\Theo{de représentation de Riesz-Fréchet}{Soit $l\in H'$, il existe un unique élément $f\in H$ tel que 
	\[\forall v\in H,\ l(v)=(l,v)_{H',H} = (f,v)_{H'}\]

et on a $\|f\|_H = \|l\|_{H'}$.}

\section{Théorème de Lax-Milgram et problème variationnel abstrait}
On considère un espace de Hilbert $V$ et $V'$ son dual. Soit $a(\bullet,\bullet)$ une fonctionnelle :
\begin{itemize}
	\item bilinéaire de $V\times V$ dans $\mathbb{R}$
	\item continue ($\exists M; \forall u,v\in V, |a(u,v)|\leq M\|u\|\|v\|$)
	\item V-elliptique ($\exists \alpha>0; a(v,v)\geq \alpha\|v\|^2$)
\end{itemize}

Soit $L\in V'$. Le problème variationnel est alors défini comme suit :
\[(PV)\left\{\begin{array}{c} \text{Chercher } u\in V \text{ tel que } \forall v\in V \\
a(u,v)=L(v) \end{array}\right. \]

\Lem{de Lax-Milgram}{Sous les hypothèses précédentes sur $a(\bullet,\bullet)$, et $L(\bullet)$, on a :
\begin{enumerate}
	\item (PV) admet une unique solution
	\item On étudie l'existence et l'unicité d'une solution du problème transformé
\end{enumerate}}

\begin{dem}
Distribuée sur feuille.
\end{dem}

\Rem{}{Si $a(\bullet,\bullet)$ est de plus symétrique, alors combiné avec la V-ellpticité, on a $a(\bullet,\bullet)$ défini positif. Donc $a(\bullet,\bullet)$ définit un produit scalaire sur V.\\
On peut donc lui associer une norme $(a(v,v))^{\frac{1}{2}}$ qui est équivalente à $\|\bullet\|_V$ (grâce à l'ellpticité et à la continuité)}

\subsection{Ecriture sous forme d'un problème de minimisation d'une fonctionnelle d'énergie}
On définit \[J:\begin{array}{c c c} V &\to& \mathbb{R} \\ v&\mapsto& J(v)=\frac{1}{2}a(v,v)-L(v) \end{array}\]
On cherche $v$ minimisant $J$

\Theo{de Stanpacchia}{Il existe un unique élément $v$ minimisant $J$, et cet élément est aussi l'unique solution de (PV)}

\begin{dem}
Soient $u,w\in V$
\begin{eqnarray*}
	J(u+w)&=&\frac{1}{2}a(u+w,u+w)-L(u+w)\\
		&=&J(u)+J(w)+a(u,w)\\
		&=&J(u)+\frac{1}{2}a(w,w)\underbrace{-L(w)+a(u,w)}_{(*)}
\end{eqnarray*}
Si $v$ solution de (PV), alors $(*)=0$.\\
De plus, si $w\in V\textbackslash\{0\}$, $a(w,w)\geq \alpha\|w\|^2_V >0$. Donc $J(u+w)\geq J(u)$.\\
$\forall t\in V$, $\exists u,w\in V$, $u+w=t$\\
	\[J(t)\geq J(u)\]
D'où \[J(u)=\min_{u\in V} J(t)\]
(Manque la réciproque)
\end{dem}

\section{Résultat d'erreur}
\begin{minipage}{0.4\linewidth}
\[(PV)\left\{\begin{array}{c}
\text{Chercher } u\in V, \forall v\in V\\
a(u,v)=L(v)
\end{array}\right.\]
\end{minipage}\hspace{0.1\linewidth}
\begin{minipage}{0.4\linewidth}
\[(PV_h)\left\{\begin{array}{c}
\text{Chercher } u_h\in V_h, \forall v_h\in V_h\\
	a(u_hv_h)=L(v_h)
\end{array}\right.\]
\end{minipage}

On cherche à déterminer l'erreur commise en passant de $V$ à $V_h$. On herche à quantifier $\|u-u_h\|$.\\
\Lem{de Céa}{Soit $u$ la solution de $(PV)$ et $u_h$ la solution de $(PV_h)$. Alors :
	\[\|u-u_h\|\leq \frac{M}{\alpha}\inf_{v_h\in V_h} \|u-v_h\|\]}

\begin{dem}
Grâce à la $V$-ellipticité :
\begin{eqnarray*}
\alpha\|u-u_h\|_V^2\leq &a&(u-u_h,u-u_h)\\
			&=&a(u-u_h, u-v_h)+\underbrace{a(u-u_h, v_h-u_h)}_{=0}
\end{eqnarray*}
car $\forall v_h\in V_h$, $v_h-u_h\in V_h$ et :
	\[\forall v_h\in V_h,\ \left\{\begin{array}{c c c}
		a(u,v_h)&=&L(v_h)\\
		a(u_h,v_h)&=&L(v_h)
	\end{array}\right. \Rightarrow a(u-u_h,v_h)=0\]
Grâce à la continuité :
	\[\alpha\|u-u_h\|_V^2\leq M\|u-u_h\|\|u-v_h\|_V\]
	\[\|u-u_h\|_V\leq \frac{M}{\alpha} \|u-v_h\|\]
Et ce pour tout $v_h\in V_h$, d'où le résultat.
\end{dem}

On suppose qu'il existe un sous espace $\mathcal{V}$ inclu et dense dans V, et une application $r_h:\mathcal{V}\to V$. (Par exemple, l'interpolé de Lagrange $\Pi_h$ vu en TD) tels que 
	\[\forall v\in \mathcal{V},\ \lim_{h\to 0} \|v-v_h\|=0\]
Alors la méthode d'approximation variationnelle converge :
	\[\lim_{h\to 0} \|u-u_h\| =0\]
(Dans la preuve, on utilise le lemme de Céa)
